\chapter{Curve Definition}
\label{curvedefinition}

This chapter describes all functions in the Curve Definition module.
\section{Interpolation}
In this section we treat different kinds of interpolation of
points or points and derivatives (Hermite). In addition to the general
functions there are functions to find fillet curves
(a curve between two other curves),
and blending curves (a curve between the end points of two other curves).
\subsection{Compute a curve interpolating a straight line between two points.}
\funclabel{s1602}
\begin{minipg1}
To make a straight line represented as a B-spline curve between two points.
\end{minipg1}\\ \\
SYNOPSIS\\
        \>void s1602(\begin{minipg3}
                {\fov startpt}, {\fov endpt}, {\fov order}, {\fov dim}, {\fov startpar}, {\fov endpar},
                {\fov curve}, {\fov stat})
                \end{minipg3}\\[0.3ex]
                \>\>    double  \>      {\fov startpt}[\,];\\
                \>\>    double  \>      {\fov endpt}[\,];\\
                \>\>    int     \>      {\fov order};\\
                \>\>    int     \>      {\fov dim};\\
                \>\>    double  \>      {\fov startpar};\\
                \>\>    double  \>      *{\fov endpar};\\
                \>\>    SISLCurve       \>      **{\fov curve};\\
                \>\>    int     \>      *{\fov stat};\\
\newpagetabs
ARGUMENTS\\
        \>Input Arguments:\\
        \>\>    {\fov startpt}  \> - \> Start point of the straight line\\
        \>\>    {\fov endpt}    \> - \> End point of the straight line\\
        \>\>    {\fov order}    \> - \> The order of the curve to be made.\\
        \>\>    {\fov dim}      \> - \> The dimension of the geometric space\\
        \>\>    {\fov startpar} \> - \> Start value of the parameterization of the curve\\
\\
        \>Output Arguments:\\
        \>\>    {\fov endpar }\> - \>   Parameter value used at the end of the curve\\
        \>\>    {\fov curve}    \> - \> Pointer to the B-spline curve\\
        \>\>    {\fov stat}     \> - \> Status messages\\
                \>\>\>\>\>              $> 0$   : warning\\
                \>\>\>\>\>              $= 0$   : ok\\
                \>\>\>\>\>              $< 0$   : error\\
\\
EXAMPLE OF USE\\
                \>      \{ \\
                \>\>    double          \>{\fov startpt}[2];\\
                \>\>    double          \>{\fov endpt}[2];\\
                \>\>    int             \>{\fov order};\\
                \>\>    int             \>{\fov dim};\\
                \>\>    double          \>{\fov startpar};\\
                \>\>    double          \>{\fov endpar};\\
                \>\>    SISLCurve               \>*{\fov curve};\\
                \>\>    int             \>{\fov stat};\\
                \>\>    \ldots \\
        \>\>s1602(\begin{minipg4}
                {\fov startpt}, {\fov endpt}, {\fov order}, {\fov dim},
        {\fov startpar}, \&{\fov endpar}, \&{\fov curve}, \&{\fov stat});
                        \end{minipg4}\\
                \>\>    \ldots \\
                \>      \} \\
\end{tabbing}

\pgsbreak
\subsection{Compute a curve interpolating a set of points,
\mbox{automatic} parameterization.}
\funclabel{s1356}
\begin{minipg1}
Compute a curve interpolating a set of points.  The points
can be assigned a tangent (derivative).  The parameterization of the
curve will be generated and the curve can be open, closed non-periodic
or periodic.  If end-conditions are conflicting, the condition closed
curve rules out other end conditions.
The output will be represented as a B-spline curve.
\end{minipg1} \\ \\
SYNOPSIS\\
        \>void s1356(\begin{minipg3}
          {\fov epoint}, {\fov inbpnt}, {\fov idim}, {\fov nptyp}, {\fov icnsta}, {\fov icnend}, {\fov iopen}, {\fov ik}, {\fov astpar},
          {\fov cendpar}, {\fov rc}, {\fov gpar}, {\fov jnbpar}, {\fov jstat})
        \end{minipg3}\\[0.3ex]
        \>\>    double \>  {\fov epoint}[\,];\\
        \>\>    int    \>  {\fov inbpnt};\\
        \>\>    int    \>  {\fov idim};\\
        \>\>    int    \>  {\fov nptyp}[\,];\\
        \>\>    int    \>  {\fov icnsta};\\
        \>\>    int    \>  {\fov icnend};\\
        \>\>    int    \>  {\fov iopen};\\
        \>\>    int    \>  {\fov ik};\\
        \>\>    double \>  {\fov astpar};\\
        \>\>    double \>  *{\fov cendpar};\\
        \>\>    SISLCurve \> **{\fov rc};\\
        \>\>    double \>  **{\fov gpar};\\
        \>\>    int    \>  *{\fov jnbpar};\\
        \>\>    int    \>  *{\fov jstat};\\
\\
ARGUMENTS\\
        \>Input Arguments:\\
        \>\>    {\fov epoint}\> - \>
        \begin{minipg2}
          Array (of length $idim\times inbpnt$) containing the
          points/\-derivatives to be interpolated.
        \end{minipg2}\\
        \>\>    {\fov inbpnt}\> - \>
        \begin{minipg2}
          No. of points/\-derivatives in the {\fov epoint} array.
        \end{minipg2}\\
        \>\>    {\fov idim}\> - \>
        The dimension of the space in which the points lie.\\
        \>\>    {\fov nptyp}\> - \> \begin{minipg2}
                  Array (length {\fov inbpnt}) containing type indicator for
                  points/\-derivatives/\-second-derivatives:
                \end{minipg2}\\
                \>\>\>\> $=1$\>: Ordinary point.\\
                \>\>\>\> $=2$\>:
                \begin{minipg5}
                  Knuckle point.  (Is treated as an ordinary point.)
                \end{minipg5}\\
                \>\>\>\> $=3$\>: Derivative to next point.\\
                \>\>\>\> $=4$\>: Derivative to prior point.\\
                \>\>\>\> ($=5$\>: Second-derivative to next point.)\\
                \>\>\>\> ($=6$\>: Second derivative to prior point.)\\
                \>\>\>\> $=13$\>: Point of tangent to next point.\\
                \>\>\>\> $=14$\>: Point of tangent to prior  point.\\
\newpagetabs
        \>\>    {\fov icnsta}\> - \>
                Additional condition at the start of the curve:\\
                  \>\>\>\> $=0$\>: No additional condition.\\
                  \>\>\>\> $=1$\>: Zero curvature at start.\\
        \>\>    {\fov icnend}\> - \>
                Additional condition at the end of the curve:\\
                  \>\>\>\> $=0$\>: No additional condition.\\
                  \>\>\>\> $=1$\>: Zero curvature at end.\\
        \>\>    {\fov iopen}\> - \>
                Flag telling if the curve should be open or closed:\\
                  \>\>\>\> $=1$\>: Open curve.\\
                  \>\>\>\> $=0$\>: Closed, non-periodic curve.\\
                  \>\>\>\> $=-1$\>: Periodic (and closed) curve.\\
        \>\>    {\fov ik}\> - \> The order of the spline curve to be produced.\\
        \>\>    {\fov astpar}\> - \>
        \begin{minipg2}
          Parameter value to be used at the start of the curve.
        \end{minipg2}\\
\\
        \>Output Arguments:\\
        \>\>    {\fov cendpar}\> - \>
        \begin{minipg2}
          Parameter value used at the end of the curve.
        \end{minipg2}\\
        \>\>    {\fov rc}\> - \> Pointer to output B-spline curve.\\
        \>\>    {\fov gpar}\> - \>
        \begin{minipg2}
          Pointer to the parameter values of the points in the
          curve. Represented only once, although derivatives and
          second-derivatives will have the same parameter value as the
          points.
        \end{minipg2}\\
        \>\>    {\fov jnbpar}\> - \> No. of unique parameter values.\\
        \>\>    {\fov jstat}\> - \> Status message\\
                \>\>\>\>\> $< 0$ : Error.\\
                \>\>\>\>\> $= 0$ : Ok.\\
                \>\>\>\>\> $> 0$ : Warning.\\
\\ %\newpagetabs
EXAMPLE OF USE\\
        \>      \{ \\
        \>\>    double \>  {\fov epoint}[30];\\
        \>\>    int    \>  {\fov inbpnt} = 10;\\
        \>\>    int    \>  {\fov idim} = 3;\\
        \>\>    int    \>  {\fov nptyp}[10];\\
        \>\>    int    \>  {\fov icnsta} = 0;\\
        \>\>    int    \>  {\fov icnend} = 0;\\
        \>\>    int    \>  {\fov iopen} = 1;\\
        \>\>    int    \>  {\fov ik} = 4;\\
        \>\>    double \>  {\fov astpar} = 0.0;\\
        \>\>    double \>  {\fov cendpar} = 0.0;\\
        \>\>    SISLCurve \> *{\fov rc} = NULL;\\
        \>\>    double \>  *{\fov gpar} = NULL;\\
        \>\>    int    \>  {\fov jnbpar} = 0;\\
        \>\>    int    \>  {\fov jstat};\\
        \>\>    \ldots \\
        \>\>s1356(\begin{minipg4}
          {\fov epoint}, {\fov inbpnt}, {\fov idim}, {\fov nptyp}, {\fov icnsta}, {\fov icnend}, {\fov iopen}, {\fov ik}, {\fov astpar},
          \&{\fov cendpar}, \&{\fov rc}, \&{\fov gpar}, \&{\fov jnbpar}, \&{\fov jstat});
        \end{minipg4}\\
        \>\>    \ldots \\
        \>      \}
\end{tabbing}

\pgsbreak
\subsection{Compute a curve interpolating a set of points,
parameter\-ization as input.}
\funclabel{s1357}
\begin{minipg1}
Compute a curve interpolating a set of points.  The points
can be assigned a tangent (derivative).  The curve can be open, closed
or periodic. If end-conditions are conflicting, the condition closed
curve rules out other end conditions. The parameterization is given by
the array {\fov epar}.
The output will be represented as a B-spline curve.
\end{minipg1} \\ \\
SYNOPSIS\\
        \>void s1357(\begin{minipg3}
        {\fov epoint}, {\fov inbpnt}, {\fov idim}, {\fov ntype}, {\fov epar}, {\fov icnsta}, {\fov icnend}, {\fov iopen}, {\fov ik}, {\fov astpar},
        {\fov cendpar}, {\fov rc}, {\fov gpar}, {\fov jnbpar}, {\fov jstat})
        \end{minipg3}\\[0.3ex]
        \>\>    double \>  {\fov epoint}[\,];\\
        \>\>    int    \>  {\fov inbpnt};\\
        \>\>    int    \>  {\fov idim};\\
        \>\>    int    \>  {\fov ntype}[\,];\\
        \>\>    double \>  {\fov epar}[\,];\\
        \>\>    int    \>  {\fov icnsta};\\
        \>\>    int    \>  {\fov icnend};\\
        \>\>    int    \>  {\fov iopen};\\
        \>\>    int    \>  {\fov ik};\\
        \>\>    double \>  {\fov astpar};\\
        \>\>    double \>  *{\fov cendpar};\\
        \>\>    SISLCurve \> **{\fov rc};\\
        \>\>    double \>  **{\fov gpar};\\
        \>\>    int    \>  *{\fov jnbpar};\\
        \>\>    int    \>  *{\fov jstat};\\
\\
ARGUMENTS\\
        \>Input Arguments:\\
        \>\>    {\fov epoint}\> - \>
        \begin{minipg2}
          Array (length $idim\times inbpnt$) containing the
          points/\-derivatives to be interpolated.
        \end{minipg2}\\
        \>\>    {\fov inbpnt}\> - \>
        \begin{minipg2}
          No. of points/\-derivatives in the {\fov epoint} array.
        \end{minipg2}\\
        \>\>    {\fov idim}\> - \>
        The dimension of the space in which the points lie.\\
        \>\>    {\fov ntype}\> - \> \begin{minipg2}
                  Array (length {\fov inbpnt}) containing type indicator for
                  points/\-derivatives/\-second-derivatives:
                \end{minipg2}\\
                \>\>\>\> $=1$\>: Ordinary point.\\
                \>\>\>\> $=2$\>:
                \begin{minipg5}
                  Knuckle point.  (Is treated as an ordinary point.)
                \end{minipg5}\\
                \>\>\>\> $=3$\>: Derivative to next point.\\
                \>\>\>\> $=4$\>: Derivative to prior point.\\
                \>\>\>\> ($=5$\>: Second-derivative to next point.)\\
                \>\>\>\> ($=6$\>: Second derivative to prior point.)\\
                \>\>\>\> $=13$\>: Point of tangent to next point.\\
                \>\>\>\> $=14$\>: Point of tangent to prior  point.\\
        \>\>    {\fov epar}\> - \>
        \begin{minipg2}
          Array containing the wanted parameterization. Only parameter
          values corresponding to position points are given.
          For closed curves, one additional parameter value
          must be specified. The last entry contains
          the parametrization of the repeated start point.
          (if the end point is equal to the start point of
          the interpolation the length of the array should
          be equal to inpt1 also in the closed case).
        \end{minipg2}\\
        \>\>    {\fov icnsta}\> - \>
                Additional condition at the start of the curve:\\
                  \>\>\>\> $=0$\>: No additional condition.\\
                  \>\>\>\> $=1$\>: Zero curvature at start.\\
        \>\>    {\fov icnend}\> - \>
                Additional condition at the end of the curve:\\
                  \>\>\>\> $=0$\>: No additional condition.\\
                  \>\>\>\> $=1$\>: Zero curvature at end.\\
        \>\>    {\fov iopen}\> - \>
                Flag telling if the curve should be open or closed:\\
                  \>\>\>\> $=1$\>: The curve should be open.\\
                  \>\>\>\> $=0$\>: The curve should be closed.\\
                  \>\>\>\> $=-1$\>: The curve should be closed and periodic.\\
        \>\>    {\fov ik}\> - \> The order of the spline curve to be produced.\\
        \>\>    {\fov astpar}\> - \>
        \begin{minipg2}
          Parameter value to be used at the start of the curve.
        \end{minipg2}\\
\\
        \>Output Arguments:\\
        \>\>    {\fov cendpar}\> - \>
        \begin{minipg2}
          Parameter value used at the end of the curve.
        \end{minipg2}\\
        \>\>    {\fov rc}\> - \> Pointer to the output B-spline curve.\\
        \>\>    {\fov gpar}\> - \>
        \begin{minipg2}
          Pointer to the parameter values of the points in the
          curve. Represented only once, although derivatives and
          second-derivatives will have the same parameter value as the
          points.
        \end{minipg2}\\
        \>\>    {\fov jnbpar}\> - \>  No, of unique parameter values.\\
        \>\>    {\fov jstat}\> - \> Status message\\
                \>\>\>\>\> $< 0$ : Error.\\
                \>\>\>\>\> $= 0$ : Ok.\\
                \>\>\>\>\> $> 0$ : Warning.\\
\newpagetabs
EXAMPLE OF USE\\
        \>      \{ \\
        \>\>    double \>  {\fov epoint}[30];\\
        \>\>    int    \>  {\fov inbpnt} = 10;\\
        \>\>    int    \>  {\fov idim} = 3;\\
        \>\>    int    \>  {\fov ntype}[10];\\
        \>\>    double \>  {\fov epar}[10];\\
        \>\>    int    \>  {\fov icnsta} = 0;\\
        \>\>    int    \>  {\fov icnend} = 0;\\
        \>\>    int    \>  {\fov iopen} = 0;\\
        \>\>    int    \>  {\fov ik} = 4;\\
        \>\>    double \>  {\fov astpar} = 0.0;\\
        \>\>    double \>  {\fov cendpar};\\
        \>\>    SISLCurve \> *{\fov rc};\\
        \>\>    double \>  *{\fov gpar};\\
        \>\>    int    \>  {\fov jnbpar};\\
        \>\>    int    \>  {\fov jstat};\\
        \>\>    \ldots \\
        \>\>s1357(\begin{minipg4}
          {\fov epoint}, {\fov inbpnt}, {\fov idim}, {\fov ntype}, {\fov epar}, {\fov icnsta}, {\fov icnend}, {\fov iopen}, {\fov ik}, {\fov astpar},
          \&{\fov cendpar}, \&{\fov rc}, \&{\fov gpar}, \&{\fov jnbpar}, \&{\fov jstat});
        \end{minipg4}\\
        \>\>    \ldots \\
        \>      \}
\end{tabbing}

\pgsbreak
\subsection{\sloppy Compute a curve by Hermite interpolation, automatic parameteriza\-tion.}
\funclabel{s1380}
\begin{minipg1}
  To compute the cubic Hermite interpolant to the data given by the points
  point and the derivatives derivate.
 The output is represented as a B-spline curve.
\end{minipg1}\\ \\
SYNOPSIS\\
        \>void s1380(\begin{minipg3}
                {\fov point}, {\fov derivate}, {\fov numpt}, {\fov dim}, {\fov typepar}, {\fov curve}, {\fov stat})
                \end{minipg3}\\
                \>\>    double  \>      {\fov point}[\,];\\
                \>\>    double  \>      {\fov derivate}[\,];\\
                \>\>    int     \>      {\fov numpt};\\
                \>\>    int     \>      {\fov dim};\\
                \>\>    int     \>      {\fov typepar};\\
                \>\>    SISLCurve       \>      **{\fov curve};\\
                \>\>    int     \>      *{\fov stat};\\
\\
ARGUMENTS\\
        \>Input Arguments:\\
        \>\>    {\fov point}    \> - \> \begin{minipg2}
                                Array (length dim*numpt) containing the
                                points in sequence
                                $(x_{0},y_{0},x_{1},y_{1},\ldots)$
                                to be interpolated.
                                \end{minipg2}\\[0.3ex]
        \>\>    {\fov derivate}\> - \>  \begin{minipg2}
                                Array (length dim*numpt) containing the
                                derivate in sequence
                                $(\frac{dx_{0}}{dt},\frac{dy_{0}}{dt},
                                \frac{dx_{1}}{dt},\frac{dy_{1}}{dt},\ldots)$
                                to be interpolated.
                                \end{minipg2}\\[0.3ex]
        \>\>    {\fov numpt}    \> - \> \begin{minipg2}
                                No. of points/derivatives in the
                                point and derivative arrays.
                                \end{minipg2}\\[0.3ex]
        \>\>    {\fov dim}      \> - \> \begin{minipg2}
                                The dimension of the space in which
                                the points lie.
                                \end{minipg2}\\
        \>\>    {\fov typepar}  \> - \>
                                Type of parameterization:\\
                \>\>\>\>\>      $=1$ : \>  \begin{minipg5}
                                Parameterization using cord length\\
                                between the points.
                                \end{minipg5}\\[0.3ex]
                \>\>\>\>\>      $\neq 1$ :\>  Uniform parameterization.\\
\\
        \>Output Arguments:\\
        \>\>    {\fov curve}    \> - \> Pointer to the output B-spline curve\\
        \>\>    {\fov stat}     \> - \> Status messages\\
                \>\>\>\>\>              $> 0$   : warning\\
                \>\>\>\>\>              $= 0$   : ok\\
                \>\>\>\>\>              $< 0$   : error\\
\newpagetabs
EXAMPLE OF USE\\
                \>      \{ \\
                \>\>    double  \>      {\fov point}[10];\\
                \>\>    double  \>      {\fov derivate}[10];\\
                \>\>    int     \>      {\fov numpt} = 5;\\
                \>\>    int     \>      {\fov dim} = 2;\\
                \>\>    int     \>      {\fov typepar};\\
                \>\>    SISLCurve       \>      *{\fov curve};\\
                \>\>    int     \>      {\fov stat};\\
                \>\>    \ldots \\
        \>\>s1380(\begin{minipg4}
                {\fov point}, {\fov derivate}, {\fov numpt}, {\fov dim}, {\fov typepar}, \&{\fov curve}, \&{\fov stat});
                        \end{minipg4}\\
                \>\>    \ldots \\
                \>      \} \\
\end{tabbing}

\pgsbreak
\subsection{Compute a curve by Hermite interpolation,
parameter\-ization as input.}
\funclabel{s1379}
\begin{minipg1}
To compute the cubic Hermite interpolant to the data given by the points
point and the derivatives derivate and the parameterization par.
The output is represented as a B-spline curve.
\end{minipg1}\\ \\
SYNOPSIS\\
        \>void s1379(\begin{minipg3}
                {\fov point}, {\fov derivate}, {\fov par}, {\fov numpt}, {\fov dim}, {\fov curve}, {\fov stat})
                \end{minipg3}\\
                \>\>    double  \>      {\fov point}[\,];\\
                \>\>    double  \>      {\fov derivate}[\,];\\
                \>\>    double  \>      {\fov par}[\,];\\
                \>\>    int     \>      {\fov numpt};\\
                \>\>    int     \>      {\fov dim};\\
                \>\>    SISLCurve       \>      **{\fov curve};\\
                \>\>    int     \>      *{\fov stat};\\
\\
ARGUMENTS\\
        \>Input Arguments:\\
        \>\>    {\fov point}    \> - \> \begin{minipg2}
                                Array (length dim*numpt) containing the
                                points to be interpolated in the sequence is
                                $(x_{0},y_{0},x_{1},y_{1},\ldots)$
                                .
                                \end{minipg2}\\[0.3ex]
        \>\>    {\fov derivate}\> - \>  \begin{minipg2}
                                Array (length dim*numpt) containing the
                                derivatives to be interpolated in the sequence is
                                \[
                                (\frac{dx_{0}}{dt},\frac{dy_{0}}{dt},
                                \frac{dx_{1}}{dt},\frac{dy_{1}}{dt},\ldots).
                                \]
                                \end{minipg2}\\[0.8ex]
        \>\>    {\fov par}      \> - \> \begin{minipg2}
                                Parameterization array,
                                $(t_{0},t_{1},\ldots)$. The array should
                                be increasing in value.
                                \end{minipg2}\\[0.3ex]
        \>\>    {\fov numpt}    \> - \> \begin{minipg2}
                                No. of points/derivatives in the
                                point and derivative arrays.
                                \end{minipg2}\\[0.3ex]
        \>\>    {\fov dim}      \> - \> \begin{minipg2}
                                The dimension of the space in which
                                the points lie.
                                \end{minipg2}\\[0.3ex]
\\
        \>Output Arguments:\\
        \>\>    {\fov curve}    \> - \> Pointer to output B-spline curve\\
        \>\>    {\fov stat}     \> - \> Status messages\\
                \>\>\>\>\>              $> 0$   : warning\\
                \>\>\>\>\>              $= 0$   : ok\\
                \>\>\>\>\>              $< 0$   : error\\
\newpagetabs
EXAMPLE OF USE\\
                \>      \{ \\
                \>\>    double  \>      {\fov point}[10];\\
                \>\>    double  \>      {\fov derivate}[10];\\
                \>\>    double  \>      {\fov par}[5];\\
                \>\>    int     \>      {\fov numpt} = 5;\\
                \>\>    int     \>      {\fov dim} = 2;\\
                \>\>    SISLCurve       \>      *{\fov curve};\\
                \>\>    int     \>      {\fov stat};\\
                \>\>    \ldots \\
        \>\>s1379(\begin{minipg4}
                {\fov point}, {\fov derivate}, {\fov par}, {\fov numpt}, {\fov dim}, \&{\fov curve}, \&{\fov stat});
                        \end{minipg4}\\
                \>\>    \ldots \\
                \>      \} \\
\end{tabbing}

\pgsbreak
\subsection{Compute a fillet curve based on parameter value.}
\funclabel{s1607}
\begin{minipg1}
  To calculate a fillet curve between two curves. The start and end point
  for the fillet is given as one parameter value for each of the
  curves.
  The output is represented as a B-spline curve.
\end{minipg1} \\ \\
SYNOPSIS\\
        \>void s1607(\begin{minipg3}
        {\fov curve1}, {\fov curve2}, {\fov epsge}, {\fov end1}, {\fov fillpar1}, {\fov end2}, {\fov fillpar2},
        {\fov filltype}, {\fov dim}, {\fov order}, {\fov newcurve}, {\fov stat})
                \end{minipg3}\\[0.3ex]
                \>\>    SISLCurve       \>      *{\fov curve1};\\
                \>\>    SISLCurve       \>      *{\fov curve2};\\
                \>\>    double  \>      {\fov epsge};\\
                \>\>    double  \>      {\fov end1};\\
                \>\>    double  \>      {\fov fillpar1};\\
                \>\>    double  \>      {\fov end2};\\
                \>\>    double  \>      {\fov fillpar2};\\
                \>\>    int     \>      {\fov filltype};\\
                \>\>    int     \>      {\fov dim};\\
                \>\>    int     \>      {\fov order};\\
                \>\>    SISLCurve       \>      **{\fov newcurve};\\
                \>\>    int     \>      *{\fov stat};\\
\\
ARGUMENTS\\
        \>Input Arguments:\\
        \>\>    {\fov curve1}   \> - \> The first input curve.\\
        \>\>    {\fov curve2}   \> - \> The second input curve.\\
        \>\>    {\fov epsge}    \> - \> Geometry resolution.\\
        \>\>    {\fov end1}     \> - \> \begin{minipg2}
                        Parameter value on the first curve. The parameter fillpar1 divides the first curve in two pieces. End1 is used to select which of these pieces the fillet should extend.
                                \end{minipg2}\\[0.8ex]
        \>\>    {\fov fillpar1}\> - \>  \begin{minipg2}
                        Parameter value of the start point of the fillet
                                on the first curve.
                                \end{minipg2}\\[0.8ex]
        \>\>    {\fov end2}     \> - \> \begin{minipg2}
                        Parameter value on the second curve indicating that
                        the part of the curve lying on this side of fillpar2
                                shall not be replaced by the fillet.
                                \end{minipg2}\\[0.3ex]
        \>\>    {\fov fillpar2}\> - \>  \begin{minipg2}
                        Parameter value of the start point of the fillet
                                on the second curve.
                                \end{minipg2}\\
\newpagetabs
        \>\>    {\fov filltype}\> - \>  Indicator of the type of fillet.\\
                \>\>\>\>\>      $=1$ : \>\begin{minipg5}
                                Circle approximation, interpolating tangent on first
                                curve, not on curve 2.
                                \end{minipg5}\\[0.3ex]
                \>\>\>\>\>      $=2$ : \>\begin{minipg5}
                                Conic approximation if possible,
                                \end{minipg5}\\
                \>\>\>\>\>      else : \>\begin{minipg5}
                                polynomial segment.
                                \end{minipg5}\\
        \>\>    {\fov dim}      \> - \> Dimension of space.\\
        \>\>    {\fov order}    \> - \> Order of the fillet curve, which is not always used.\\
\\
        \>Output Arguments:\\
        \>\>    {\fov newcurve}\> - \> Pointer to the B-spline fillet curve.\\
        \>\>    {\fov stat}     \> - \> Status messages\\
                \>\>\>\>\>              $> 0$   : warning\\
                \>\>\>\>\>              $= 0$   : ok\\
                \>\>\>\>\>              $< 0$   : error\\
\\
EXAMPLE OF USE\\
                \>      \{ \\
                \>\>    SISLCurve       \>      *{\fov curve1};\\
                \>\>    SISLCurve       \>      *{\fov curve2};\\
                \>\>    double  \>      {\fov epsge};\\
                \>\>    double  \>      {\fov end1};\\
                \>\>    double  \>      {\fov fillpar1};\\
                \>\>    double  \>      {\fov end2};\\
                \>\>    double  \>      {\fov fillpar2};\\
                \>\>    int     \>      {\fov filltype};\\
                \>\>    int     \>      {\fov dim};\\
                \>\>    int     \>      {\fov order};\\
                \>\>    SISLCurve       \>      *{\fov newcurve};\\
                \>\>    int     \>      {\fov stat};\\
                \>\>    \ldots \\
        \>\>s1607(\begin{minipg4}
        {\fov curve1}, {\fov curve2}, {\fov epsge}, {\fov end1}, {\fov fillpar1}, {\fov end2}, {\fov fillpar2},
        {\fov filltype}, {\fov dim}, {\fov order}, \&{\fov newcurve}, \&{\fov stat});
                        \end{minipg4}\\
                \>\>    \ldots \\
                \>      \} \\
\end{tabbing}

\pgsbreak
\subsection{Compute a fillet curve based on points.}
\funclabel{s1608}
\begin{minipg1}
  To calculate a fillet curve between two curves. Points indicate
  between which points on the input curve the fillet is to be produced.
  The output is represented as a B-spline curve.
\end{minipg1} \\ \\
SYNOPSIS\\
        \>void s1608(\begin{minipg3}
        {\fov curve1}, {\fov curve2}, {\fov epsge}, {\fov point1}, {\fov startpt1}, {\fov point2}, {\fov endpt2},
        {\fov filltype}, {\fov dim}, {\fov order}, {\fov newcurve}, {\fov parpt1}, {\fov parspt1}, {\fov parpt2},
        {\fov parept2}, {\fov stat})
                \end{minipg3}\\[0.3ex]
                \>\>    SISLCurve       \>      *{\fov curve1};\\
                \>\>    SISLCurve       \>      *{\fov curve2};\\
                \>\>    double  \>      {\fov epsge};\\
                \>\>    double  \>      {\fov point1}[\,];\\
                \>\>    double  \>      {\fov startpt1}[\,];\\
                \>\>    double  \>      {\fov point2}[\,];\\
                \>\>    double  \>      {\fov endpt2}[\,];\\
                \>\>    int     \>      {\fov filltype};\\
                \>\>    int     \>      {\fov dim};\\
                \>\>    int     \>      {\fov order};\\
                \>\>    SISLCurve       \>      **{\fov newcurve};\\
                \>\>    double  \>      *{\fov parpt1};\\
                \>\>    double  \>      *{\fov parspt1};\\
                \>\>    double  \>      *{\fov parpt2};\\
                \>\>    double  \>      *{\fov parept2};\\
                \>\>    int     \>      *{\fov stat};\\
\\
ARGUMENTS\\
        \>Input Arguments:\\
        \>\>    {\fov curve1}   \> - \> The first input curve.\\
        \>\>    {\fov curve2}   \> - \> The second input curve.\\
        \>\>    {\fov epsge}    \> - \> Geometry resolution.\\
        \>\>    {\fov point1}   \> - \> \begin{minipg2}
                        Point close to curve 1 indicating that the part of the
                        curve lying on this side of startpt1 is
                        not to be replaced by the fillet.
                                \end{minipg2}\\[0.3ex]
        \>\>    {\fov startpt1}\> - \>  \begin{minipg2}
                        Point close to curve 1, indicating where the fillet is
                        to start. The tangent at the start of the fillet will
                        have the same orientation as the curve
                        from point1 to startpt1.
                                \end{minipg2}\\[0.3ex]
        \>\>    {\fov point2}   \> - \> \begin{minipg2}
                        Point close to curve 2 indicating that the part of the
                        curve lying on this side of endpt2 is not
                        to be replaced by the fillet.
                                \end{minipg2}\\[0.8ex]
        \>\>    {\fov endpt2}   \> - \> \begin{minipg2}
                        Point close to curve two, indicating where the fillet
                        is to end. The tangent at the end of the fillet will
                        have the same orientation as the curve
                        from endpt2 to point2.
                                \end{minipg2}\\[0.3ex]
\newpagetabs
        \>\>    {\fov filltype}\> - \>  Indicator of type of fillet.\\
                \>\>\>\>\>      $=1$ : \>\begin{minipg5}
                                Circle, interpolating tangent on first
                                curve, not on curve 2.
                                \end{minipg5}\\[0.3ex]
                \>\>\>\>\>      $=2$ : \>\begin{minipg5}
                                Conic if possible,
                                \end{minipg5}\\
                \>\>\>\>\>      else : \>\begin{minipg5}
                                polynomial segment.
                                \end{minipg5}\\
        \>\>    {\fov dim}      \> - \> Dimension of space.\\
        \>\>    {\fov order}    \> - \> Order of fillet curve, which is not always used.\\
\\
        \>Output Arguments:\\
        \>\>    {\fov newcurve}\> - \> Pointer to the B-spline fillet curve.\\
        \>\>    {\fov parpt1}   \> - \> \begin{minipg2}
                        Parameter value of point {\fov point1} on curve 1.
                                \end{minipg2}\\
        \>\>    {\fov parspt1}  \> - \> \begin{minipg2}
                        Parameter value of point {\fov startpt1} on curve 1.
                                \end{minipg2}\\
        \>\>    {\fov parpt2}   \> - \> \begin{minipg2}
                        Parameter value of point {\fov point2} on curve 2.
                                \end{minipg2}\\
        \>\>    {\fov parept2}  \> - \> \begin{minipg2}
                        Parameter value of point {\fov endpt2} on curve 2.
                                \end{minipg2}\\
        \>\>    {\fov stat}     \> - \> Status messages\\
                \>\>\>\>\>              $> 0$   : warning\\
                \>\>\>\>\>              $= 0$   : ok\\
                \>\>\>\>\>              $< 0$   : error\\
\\
EXAMPLE OF USE\\
                \>      \{ \\
                \>\>    SISLCurve       \>      *{\fov curve1};\\
                \>\>    SISLCurve       \>      *{\fov curve2};\\
                \>\>    double  \>      {\fov epsge};\\
                \>\>    double  \>      {\fov point1}[3];\\
                \>\>    double  \>      {\fov startpt1}[3];\\
                \>\>    double  \>      {\fov point2}[3];\\
                \>\>    double  \>      {\fov endpt2}[3];\\
                \>\>    int     \>      {\fov filltype};\\
                \>\>    int     \>      {\fov dim} = 3;\\
                \>\>    int     \>      {\fov order};\\
                \>\>    SISLCurve       \>      *{\fov newcurve};\\
                \>\>    double  \>      {\fov parpt1};\\
                \>\>    double  \>      {\fov parspt1};\\
                \>\>    double  \>      {\fov parpt2};\\
                \>\>    double  \>      {\fov parept2};\\
                \>\>    int     \>      {\fov stat};\\
                \>\>    \ldots \\
        \>\>s1608(\begin{minipg4}
        {\fov curve1}, {\fov curve2}, {\fov epsge}, {\fov point1}, {\fov startpt1}, {\fov point2}, {\fov endpt2},\linebreak[4]
        {\fov filltype}, {\fov dim}, {\fov order}, \&{\fov newcurve}, \&{\fov parpt1}, \&{\fov parspt1},
        \linebreak[4] \&{\fov parpt2}, \&{\fov parept2}, \&{\fov stat});
                        \end{minipg4}\\
                \>\>    \ldots \\
                \>      \} \\
\end{tabbing}

\pgsbreak
\subsection{Compute a fillet curve based on radius.}
\funclabel{s1609}
\begin{minipg1}
  To calculate a constant radius fillet curve between two curves
  if possible.
  The output is represented as a B-spline curve.
\end{minipg1} \\ \\
SYNOPSIS\\
        \>void s1609(\begin{minipg3}
        {\fov curve1}, {\fov curve2}, {\fov epsge}, {\fov point1}, {\fov pointf}, {\fov point2}, {\fov radius}, {\fov normal},\linebreak
        {\fov filltype}, {\fov dim}, {\fov order}, {\fov newcurve}, {\fov parend1}, {\fov parspt1}, {\fov parend2},\linebreak
        {\fov parept2}, {\fov stat})
                \end{minipg3}\\[0.3ex]
                \>\>    SISLCurve       \>      *{\fov curve1};\\
                \>\>    SISLCurve       \>      *{\fov curve2};\\
                \>\>    double  \>      {\fov epsge};\\
                \>\>    double  \>      {\fov point1}[\,];\\
                \>\>    double  \>      {\fov pointf}[\,];\\
                \>\>    double  \>      {\fov point2}[\,];\\
                \>\>    double  \>      {\fov radius};\\
                \>\>    double  \>      {\fov normal}[\,];\\
                \>\>    int     \>      {\fov filltype};\\
                \>\>    int     \>      {\fov dim};\\
                \>\>    int     \>      {\fov order};\\
                \>\>    SISLCurve       \>      **{\fov newcurve};\\
                \>\>    double  \>      *{\fov parend1};\\
                \>\>    double  \>      *{\fov parspt1};\\
                \>\>    double  \>      *{\fov parend2};\\
                \>\>    double  \>      *{\fov parept2};\\
                \>\>    int     \>      *{\fov stat};\\
\\
ARGUMENTS\\
        \>Input Arguments:\\
        \>\>    {\fov curve1}   \> - \> The first input curve.\\
        \>\>    {\fov curve2}   \> - \> The second input curve.\\
        \>\>    {\fov epsge}    \> - \> Geometry resolution.\\
        \>\>    {\fov point1}   \> - \> \begin{minipg2}
                        Point indicating that the fillet should be put on the
                        side of {\fov curve1} where {\fov point1} is situated.
                                \end{minipg2}\\[0.3ex]
        \>\>    {\fov pointf}\> - \>    \begin{minipg2}
                        Point indicating where the fillet curve should go.
                        {\fov point1} together with {\fov pointf}
                        indicates the direction of the start tangent of
                        the curve, while pointf together with {\fov point2}
                        indicates the direction of the end tangent of
                        the curve.  If more than one position of the fillet
                        curve is possible, the closest curve to {\fov pointf}
                        is chosen.
                                \end{minipg2}\\[0.3ex]
        \>\>    {\fov point2}   \> - \> \begin{minipg2}
                        Point indicating that the fillet should be put on the
                        side of {\fov curve2} where {\fov point2} is situated.
                                \end{minipg2}\\[0.3ex]
        \>\>    {\fov radius}   \> - \> \begin{minipg2}
                The radius to be used on the fillet if a circular fillet is
                possible, otherwise a conic or a quadratic polynomial
                curve is used, approximating the circular fillet.
                                \end{minipg2}\\[0.3ex]
        \>\>    {\fov normal}   \> - \> \begin{minipg2}
                        Normal to the plane the fillet curve
                        should lie close to. This is only used in 3D
                        fillet calculations, and the fillet centre will
                        be in the direction of the cross product of the
                        curve tangents and the normal.
                                \end{minipg2}\\[0.3ex]
        \>\>    {\fov filltype}\> - \>  Indicator of type of fillet.\\
                \>\>\>\>\>      $=1$ : \>\begin{minipg5}
                                Circle, interpolating tangent on first
                                curve, not on curve 2.
                                \end{minipg5}\\[0.3ex]
                \>\>\>\>\>      $=2$ : \>\begin{minipg5}
                                Conic if possible,
                                \end{minipg5}\\
                \>\>\>\>\>      else : \>\begin{minipg5}
                                polynomial segment.
                                \end{minipg5}\\
        \>\>    {\fov dim}      \> - \> Dimension of space.\\
        \>\>    {\fov order}    \> - \> Order of fillet curve, which is not always used.\\
\\
        \>Output Arguments:\\
        \>\>    {\fov newcurve}\> - \> Pointer to the B-spline fillet curve.\\
        \>\>    {\fov parend1}  \> - \> \begin{minipg2}
                        Parameter value of the end of curve 1 not affected
                        by the fillet.
                                \end{minipg2}\\[0.8ex]
        \>\>    {\fov parspt1}  \> - \> \begin{minipg2}
                        Parameter value of the point on curve 1 where the
                        fillet starts.
                                \end{minipg2}\\[0.8ex]
        \>\>    {\fov parend2}  \> - \> \begin{minipg2}
                        Parameter value of the end of curve 2 not affected
                        by the fillet.
                                \end{minipg2}\\[0.8ex]
        \>\>    {\fov parept2}  \> - \> \begin{minipg2}
                        Parameter value of the point on curve 2 where the
                        fillet ends.
                                \end{minipg2}\\[0.8ex]
        \>\>    {\fov stat}     \> - \> Status messages\\
                \>\>\>\>\>              $> 0$   : warning\\
                \>\>\>\>\>              $= 0$   : ok\\
                \>\>\>\>\>              $< 0$   : error\\
\newpagetabs
EXAMPLE OF USE\\
                \>      \{ \\
                \>\>    SISLCurve       \>      *{\fov curve1};\\
                \>\>    SISLCurve       \>      *{\fov curve2};\\
                \>\>    double  \>      {\fov epsge};\\
                \>\>    double  \>      {\fov point1}[3];\\
                \>\>    double  \>      {\fov pointf}[3];\\
                \>\>    double  \>      {\fov point2}[3];\\
                \>\>    double  \>      {\fov radius};\\
                \>\>    double  \>      {\fov normal}[3];\\
                \>\>    int     \>      {\fov filltype};\\
                \>\>    int     \>      {\fov dim} = 3;\\
                \>\>    int     \>      {\fov order};\\
                \>\>    SISLCurve       \>      *{\fov newcurve};\\
                \>\>    double  \>      {\fov parend1};\\
                \>\>    double  \>      {\fov parspt1};\\
                \>\>    double  \>      {\fov parend2};\\
                \>\>    double  \>      {\fov parept2};\\
                \>\>    int     \>      {\fov stat};\\
                \>\>    \ldots \\
        \>\>s1609(\begin{minipg4}
        {\fov curve1}, {\fov curve2}, {\fov epsge}, {\fov point1}, {\fov pointf}, {\fov point2}, {\fov radius},\\ {\fov normal},
        {\fov filltype}, {\fov dim}, {\fov order}, \&{\fov newcurve}, \&{\fov parend1}, \&{\fov parspt1},\\ \&{\fov parend2},
        \&{\fov parept2}, \&{\fov stat});
                        \end{minipg4}\\
                \>\>    \ldots \\
                \>      \} \\
\end{tabbing}

\pgsbreak
\subsection{Compute a circular fillet between a 2D curve and a circle.}
\funclabel{s1014}
\begin{minipg1}
  Compute the fillet by iterating to the start and end points of a
  fillet between a 2D curve and a circle. The centre of the circular
  fillet is also calculated.
\end{minipg1} \\ \\
SYNOPSIS\\
        \>void s1014(\begin{minipg3}
        {\fov pc1}, {\fov circ\_cen}, {\fov circ\_rad}, {\fov aepsge}, {\fov eps1}, {\fov eps2}, {\fov aradius},
        {\fov parpt1}, {\fov parpt2}, {\fov centre}, {\fov jstat})
        \end{minipg3}\\[0.3ex]
        \>\>    SISLCurve \> *{\fov pc1};\\
        \>\>    double    \> {\fov circ\_cen}[\,];\\
        \>\>    double    \> {\fov circ\_rad};\\
        \>\>    double    \> {\fov aepsge};\\
        \>\>    double    \> {\fov eps1}[\,];\\
        \>\>    double    \> {\fov eps2}[\,];\\
        \>\>    double    \> {\fov aradius};\\
        \>\>    double    \> *{\fov parpt1};\\
        \>\>    double    \> *{\fov parpt2};\\
        \>\>    double    \> {\fov centre}[\,];\\
        \>\>    int       \> *{\fov jstat};\\
\\
ARGUMENTS\\
        \>Input Arguments:\\
        \>\>    {\fov pc1}    \> - \> The first input curve.\\
        \>\>    {\fov circ\_cen}   \> - \> 2D centre of the circle.\\
        \>\>    {\fov circ\_rad}   \> - \> Radius of the circle.\\
        \>\>    {\fov aepsge} \> - \> Geometry resolution.\\
        \>\>    {\fov eps1}   \> - \> \begin{minipg2}
                                        2D point telling that the fillet
                                        should be put on the side of
                                        curve 1 where {\fov eps1} is situated.
                                      \end{minipg2}\\[0.8ex]
        \>\>    {\fov eps2}   \> - \> \begin{minipg2}
                                        2D point telling that the fillet
                                        should be put on the side of the
                                        input circle where {\fov eps2} is
                                        situated.
                                      \end{minipg2}\\[0.8ex]
        \>\>    {\fov aradius} \> - \> The radius to be used on the fillet.\\
\\
        \>Input/Output Arguments:\\
        \>\>    {\fov parpt1} \> - \> \begin{minipg2}
                                        Parameter value of the point on
                                        curve 1 where the fillet
                                        starts. Input is a guess value
                                        for the iteration.
                                      \end{minipg2}\\[0.8ex]
        \>\>    {\fov parpt2} \> - \> \begin{minipg2}
                                        Parameter value of the point on
                                        the input circle where the
                                        fillet ends. Input is a guess
                                        value for the iteration.
                                      \end{minipg2}\\[0.8ex]
\\
        \>Output Arguments:\\
        \>\>    {\fov centre} \> - \> \begin{minipg2}
                                        2D centre of the circular
                                        fillet.  Space must be allocated
                                        outside the function.
                                      \end{minipg2}\\[0.8ex]
        \>\>    {\fov jstat} \> - \> Status message\\
                \>\>\>\>\> $= 1$      : Converged,\\
                \>\>\>\>\> $= 2$      : Diverged,\\
                \>\>\>\>\> $< 0$      : Error.\\
\\
EXAMPLE OF USE\\
        \>      \{ \\
        \>\>    SISLCurve \> *{\fov pc1};\\
        \>\>    double    \> {\fov circ\_cen}[2];\\
        \>\>    double    \> {\fov circ\_rad};\\
        \>\>    double    \> {\fov aepsge};\\
        \>\>    double    \> {\fov eps1}[2];\\
        \>\>    double    \> {\fov eps2}[2];\\
        \>\>    double    \> {\fov aradius};\\
        \>\>    double    \> {\fov parpt1};\\
        \>\>    double    \> {\fov parpt2};\\
        \>\>    double    \> {\fov centre}[2];\\
        \>\>    int       \> {\fov jstat};\\
        \>\>    \ldots \\
        \>\>s1014(\begin{minipg4}
        {\fov pc1}, {\fov circ\_cen}, {\fov circ\_rad}, {\fov aepsge}, {\fov eps1}, {\fov eps2}, {\fov aradius},
        \&{\fov parpt1}, \&{\fov parpt2}, {\fov centre}, \&{\fov jstat});
        \end{minipg4}\\
        \>\>    \ldots \\
        \>      \}
\end{tabbing}

\pgsbreak
\subsection{Compute a circular fillet between two 2D curves.}
\funclabel{s1015}
\begin{minipg1}
  Compute the fillet by iterating to the start and end points of a
  fillet between two 2D curves. The centre of the circular fillet is
  also calculated.
\end{minipg1} \\ \\
SYNOPSIS\\
        \>void s1015(\begin{minipg3}
          {\fov pc1}, {\fov pc2}, {\fov aepsge}, {\fov eps1}, {\fov eps2}, {\fov aradius}, {\fov parpt1}, {\fov parpt2},
          {\fov centre}, {\fov jstat})
        \end{minipg3}\\[0.3ex]
                \>\>    SISLCurve  \>  *{\fov pc1};\\
                \>\>    SISLCurve  \>  *{\fov pc2};\\
                \>\>    double     \>  {\fov aepsge};\\
                \>\>    double     \>  {\fov eps1}[\,];\\
                \>\>    double     \>  {\fov eps2}[\,];\\
                \>\>    double     \>  {\fov aradius};\\
                \>\>    double     \>  *{\fov parpt1};\\
                \>\>    double     \>  *{\fov parpt2};\\
                \>\>    double     \>  {\fov centre}[\,];\\
                \>\>    int        \>  *{\fov jstat};\\
\\
ARGUMENTS\\
        \>Input Arguments:\\
        \>\>    {\fov pc1}     \> - \> The first 2D input curve.\\
        \>\>    {\fov pc2}     \> - \> The second 2D input curve.\\
        \>\>    {\fov aepsge}  \> - \> Geometry resolution.\\
        \>\>    {\fov eps1}    \> - \>
        \begin{minipg2}
          2D point telling that the fillet should be put on
          the side of curve 1 where {\fov eps1} is situated.
        \end{minipg2}\\[0.8ex]
        \>\>    {\fov eps2}    \> - \>
        \begin{minipg2}
          2D point telling that the fillet should be put on the
          side of curve 2 where {\fov eps2} is situated.
        \end{minipg2}\\[0.8ex]
        \>\>    {\fov aradius} \> - \> The radius to be used on the fillet.\\
\\
        \>Input/Output Arguments:\\
        \>\>    {\fov parpt1}  \> - \>
        \begin{minipg2}
          Parameter value of the point on curve 1 where the
          fillet starts. Input is a guess value for the iteration.
        \end{minipg2}\\[0.8ex]
        \>\>    {\fov parpt2}  \> - \>
        \begin{minipg2}
          Parameter value of the point on curve 2 where the
          fillet ends. Input is a guess value for the iteration.
        \end{minipg2}\\[0.8ex]
\\
        \>Output Arguments:\\
        \>\>    {\fov centre}  \> - \>
        \begin{minipg2}
          2D centre of the circular fillet.  Space must be
          allocated outside the function.
        \end{minipg2}\\[0.8ex]
        \>\>    {\fov jstat} \> - \> Status message\\
                \>\>\>\>\> $= 1$      : Converged,\\
                \>\>\>\>\> $= 2$      : Diverged,\\
                \>\>\>\>\> $< 0$      : Error.\\
\newpagetabs
EXAMPLE OF USE\\
        \>      \{ \\
        \>\>    SISLCurve \> *{\fov pc1};\\
        \>\>    SISLCurve \> *{\fov pc2};\\
        \>\>    double    \> {\fov aepsge};\\
        \>\>    double    \> {\fov eps1}[2];\\
        \>\>    double    \> {\fov eps2}[2];\\
        \>\>    double    \> {\fov aradius};\\
        \>\>    double    \> {\fov parpt1};\\
        \>\>    double    \> {\fov parpt2};\\
        \>\>    double    \> {\fov centre}[2];\\
        \>\>    int       \> {\fov jstat};\\
        \>\>    \ldots \\
        \>\>s1015(\begin{minipg4}
          {\fov pc1}, {\fov pc2}, {\fov aepsge}, {\fov eps1}, {\fov eps2}, {\fov aradius}, \&{\fov parpt1}, \&{\fov parpt2},
          {\fov centre}, \&{\fov jstat});
        \end{minipg4}\\
        \>\>    \ldots \\
        \>      \}
\end{tabbing}

\pgsbreak
\subsection{Compute a circular fillet between a 2D curve and a 2D line.}
\funclabel{s1016}
\begin{minipg1}
  Compute the fillet by iterating to the start and end points of a
  fillet between a 2D curve and a 2D line. The centre of the circular
  fillet is also calculated.
\end{minipg1}\\ \\
SYNOPSIS\\
        \>void s1016(\begin{minipg3}
          {\fov pc1}, {\fov point}, {\fov normal}, {\fov aepsge}, {\fov eps1}, {\fov eps2}, {\fov aradius},
          {\fov parpt1}, {\fov parpt2}, {\fov centre}, {\fov jstat})
        \end{minipg3}\\[0.3ex]
                \>\>    SISLCurve \> *{\fov pc1};\\
                \>\>    double    \> {\fov point}[\,];\\
                \>\>    double    \> {\fov normal}[\,];\\
                \>\>    double    \> {\fov aepsge};\\
                \>\>    double    \> {\fov eps1}[\,];\\
                \>\>    double    \> {\fov eps2}[\,];\\
                \>\>    double    \> {\fov aradius};\\
                \>\>    double    \> *{\fov parpt1};\\
                \>\>    double    \> *{\fov parpt2};\\
                \>\>    double    \> {\fov centre}[\,];\\
                \>\>    int       \> *{\fov jstat};\\
\\
ARGUMENTS\\
        \>Input Arguments:\\
        \>\>    {\fov pc1}     \> - \> The 2D input curve.\\
        \>\>    {\fov point}   \> - \> 2D point on the line.\\
        \>\>    {\fov normal}  \> - \> 2D normal to the line.\\
        \>\>    {\fov aepsge}  \> - \> Geometry resolution.\\
        \>\>    {\fov eps1}    \> - \>
        \begin{minipg2}
          2D point telling that the fillet should be put on
          the side of curve 1 where {\fov eps1} is situated.
        \end{minipg2}\\[0.8ex]
        \>\>    {\fov eps2}    \> - \>
        \begin{minipg2}
          2D point telling that the fillet should be put on the
          side of curve 2 where {\fov eps2} is situated.
        \end{minipg2}\\[0.8ex]
        \>\>    {\fov aradius} \> - \> The radius to be used on the fillet.\\
\\
        \>Input/Output Arguments:\\
        \>\>    {\fov parpt1}  \> - \>
        \begin{minipg2}
          Parameter value of the point on curve 1 where the
          fillet starts. Input is a guess value for the iteration.
        \end{minipg2}\\[0.8ex]
        \>\>    {\fov parpt2}  \> - \>
        \begin{minipg2}
          Parameter value of the point on the line where the
          fillet ends. Input is a guess value for the iteration.
        \end{minipg2}\\[0.8ex]
\\
        \>Output Arguments:\\
        \>\>    {\fov centre}  \> - \>
        \begin{minipg2}
          2D centre of the (circular) fillet.  Space must be
          allocated outside the function.
        \end{minipg2}\\[0.8ex]
\newpagetabs
        \>\>    {\fov jstat} \> - \> Status message\\
                \>\>\>\>\> $= 1$      : Converged,\\
                \>\>\>\>\> $= 2$      : Diverged,\\
                \>\>\>\>\> $< 0$      : Error.\\
\\
EXAMPLE OF USE\\
        \>      \{ \\
        \>\>    SISLCurve \> *{\fov pc1};\\
        \>\>    double    \> {\fov point}[2];\\
        \>\>    double    \> {\fov normal}[2];\\
        \>\>    double    \> {\fov aepsge};\\
        \>\>    double    \> {\fov eps1}[2];\\
        \>\>    double    \> {\fov eps2}[2];\\
        \>\>    double    \> {\fov aradius};\\
        \>\>    double    \> {\fov parpt1};\\
        \>\>    double    \> {\fov parpt2};\\
        \>\>    double    \> {\fov centre}[2];\\
        \>\>    int       \> {\fov jstat};\\
        \>\>    \ldots \\
        \>\>s1016(\begin{minipg4}
          {\fov pc1}, {\fov point}, {\fov normal}, {\fov aepsge}, {\fov eps1}, {\fov eps2}, {\fov aradius},
          \&{\fov parpt1}, \&{\fov parpt2}, {\fov centre}, \&{\fov jstat});
        \end{minipg4}\\
        \>\>    \ldots \\
        \>      \}
\end{tabbing}

\pgsbreak
\subsection{Compute a blending curve between two curves.}
\funclabel{s1606}
\begin{minipg1}
  To compute a blending curve between two curves.
  Two points indicate between which
  ends the blend is to be produced.
  The blending curve is either a circle or
  an approximated conic section if this is
  possible, otherwise it is a quadratic polynomial spline curve.
  The output is represented as a B-spline curve.
\end{minipg1} \\ \\
SYNOPSIS\\
        \>void s1606(\begin{minipg3}
        {\fov curve1}, {\fov curve2}, {\fov epsge}, {\fov point1}, {\fov point2},
        {\fov blendtype}, {\fov dim}, {\fov order}, {\fov newcurve}, {\fov stat})
                \end{minipg3}\\[0.3ex]
                \>\>    SISLCurve       \>      *{\fov curve1};\\
                \>\>    SISLCurve       \>      *{\fov curve2};\\
                \>\>    double  \>      {\fov epsge};\\
                \>\>    double  \>      {\fov point1}[\,];\\
                \>\>    double  \>      {\fov point2}[\,];\\
                \>\>    int     \>      {\fov blendtype};\\
                \>\>    int     \>      {\fov dim;}\\
                \>\>    int     \>      {\fov order};\\
                \>\>    SISLCurve       \>      **{\fov newcurve};\\
                \>\>    int     \>      *{\fov stat};\\
\\
ARGUMENTS\\
        \>Input Arguments:\\
        \>\>    {\fov curve1}   \> - \> The first input curve.\\
        \>\>    {\fov curve2}   \> - \> The second input curve.\\
        \>\>    {\fov epsge}    \> - \> Geometry resolution.\\
        \>\>    {\fov point1}   \> - \> \begin{minipg2}
                        Point near the end of curve 1 where the blend starts.
                                \end{minipg2}\\
        \>\>    {\fov point2}   \> - \> \begin{minipg2}
                        Point near the end of curve 2 where the blend starts.
                                \end{minipg2}\\
        \>\>    {\fov blendtype}\> - \> Indicator of type of blending.\\
                \>\>\>\>\>      $=1$ : \>\begin{minipg5}
                                Circle, interpolating tangent on first
                                curve, not on curve 2, if possible.
                                \end{minipg5}\\[0.3ex]
                \>\>\>\>\>      $=2$ : \>\begin{minipg5}
                                Conic if possible,
                                \end{minipg5}\\
                \>\>\>\>\>      else : \>\begin{minipg5}
                                polynomial segment.
                                \end{minipg5}\\
        \>\>    {\fov dim}      \> - \> Dimension of the geometry space.\\
        \>\>    {\fov order}    \> - \> Order of the blending curve.\\
\\
        \>Output Arguments:\\
        \>\>    {\fov newcurve}\> - \> Pointer to the B-spline blending curve.\\
        \>\>    {\fov stat}     \> - \> Status messages\\
                \>\>\>\>\>              $> 0$   : warning\\
                \>\>\>\>\>              $= 0$   : ok\\
                \>\>\>\>\>              $< 0$   : error\\
\newpagetabs
EXAMPLE OF USE\\
                \>      \{ \\
                \>\>    SISLCurve       \>      *{\fov curve1};\\
                \>\>    SISLCurve       \>      *{\fov curve2};\\
                \>\>    double  \>      {\fov epsge};\\
                \>\>    double  \>      {\fov point1}[3];\\
                \>\>    double  \>      {\fov point2}[3];\\
                \>\>    int     \>      {\fov blendtype};\\
                \>\>    int     \>      {\fov dim} = 3;\\
                \>\>    int     \>      {\fov order};\\
                \>\>    SISLCurve       \>      *{\fov newcurve};\\
                \>\>    int     \>      {\fov stat};\\
                \>\>    \ldots \\
        \>\>s1606(\begin{minipg4}
        {\fov curve1}, {\fov curve2}, {\fov epsge}, {\fov point1}, {\fov point2},
        {\fov blendtype}, {\fov dim}, {\fov order}, \&{\fov newcurve}, \&{\fov stat});
                        \end{minipg4}\\
                \>\>    \ldots \\
                \>      \} \\
\end{tabbing}

\pgsbreak
\section{Approximation}
Two kinds of curves are treated in this section.
The first is approximations of special shapes like
circles and conic segments.
The second is approximation of a point set, or offsets to curves.

Except for the point set approximation function, all functions
require a tolerance for the approximation.
Note that there is a close relationship
between the size of the tolerance and the amount of data
for the curve.
\subsection{Approximate a circular arc with a curve.}
\funclabel{s1303}
\begin{minipg1}
  To create a curve approximating a circular arc around
  the axis defined by the centre point, an axis vector,
  a start point and a rotational angle.
  The maximal deviation between the true circular arc and the
  approximation to the arc is controlled by the geometric
  tolerance (epsge).
  The output will be represented as a B-spline curve.
\end{minipg1} \\ \\
SYNOPSIS\\
        \>void s1303(\begin{minipg3}
                {\fov startpt}, {\fov epsge}, {\fov angle}, {\fov centrept}, {\fov axis}, {\fov dim},
                {\fov curve}, {\fov stat})
                \end{minipg3}\\[0.3ex]
                \>\>    double  \>      {\fov startpt}[\,];\\
                \>\>    double  \>      {\fov epsge};\\
                \>\>    double  \>      {\fov angle};\\
                \>\>    double  \>      {\fov centrept}[\,];\\
                \>\>    double  \>      {\fov axis}[\,];\\
                \>\>    int     \>      {\fov dim};\\
                \>\>    SISLCurve       \>      **{\fov curve};\\
                \>\>    int     \>      *{\fov stat};\\
\\
ARGUMENTS\\
        \>Input Arguments:\\
        \>\>    {\fov startpt}  \> - \> Start point of the circular arc\\
        \>\>    {\fov epsge}    \> - \> \begin{minipg2}
                                Maximal deviation allowed between the true
                                circle and the circle approximation.
                                \end{minipg2}\\[0.3ex]
        \>\>    {\fov angle}    \> - \> \begin{minipg2}
                                The rotational angle. Counterclockwise around
                                axis. If the rotational angle
                                is outside $<-2\pi,+2\pi>$
                                then a closed curve is produced.
                                \end{minipg2}\\[0.3ex]
        \>\>    {\fov centrept}\> - \>  Point on the axis of the circle.\\
        \>\>    {\fov axis}     \> - \> \begin{minipg2}
                                Normal vector to plane in which the circle lies.
                                Used if dim = 3.
                                \end{minipg2}\\[0.8ex]
        \>\>    {\fov dim}      \> - \> \begin{minipg2}
                                The dimension of the space in which the
                                circular arc lies (2 or 3).
                                \end{minipg2}\\[0.3ex]
\\
\newpagetabs
        \>Output Arguments:\\
        \>\>    {\fov curve}    \> - \> Pointer to the B-spline curve.\\
        \>\>    {\fov stat}     \> - \> Status messages\\
                \>\>\>\>\>              $> 0$   : warning\\
                \>\>\>\>\>              $= 0$   : ok\\
                \>\>\>\>\>              $< 0$   : error\\
\\
EXAMPLE OF USE\\
                \>      \{ \\
                \>\>    double  \>      {\fov startpt}[3];\\
                \>\>    double  \>      {\fov epsge};\\
                \>\>    double  \>      {\fov angle};\\
                \>\>    double  \>      {\fov centrept}[3];\\
                \>\>    double  \>      {\fov axis}[3];\\
                \>\>    int     \>      {\fov dim} = 3;\\
                \>\>    SISLCurve       \>      *{\fov curve};\\
                \>\>    int     \>      {\fov stat};\\
                \>\>    \ldots \\
        \>\>s1303(\begin{minipg4}
                        {\fov startpt}, {\fov epsge}, {\fov angle}, {\fov centrept},
                        {\fov axis}, {\fov dim}, \&{\fov curve}, \&{\fov stat});
                \end{minipg4}\\
                \>\>    \ldots \\
                \>      \} \\
\end{tabbing}

\pgsbreak
\subsection{Approximate a conic arc with a curve.}
\funclabel{s1611}
\begin{minipg1}
  To approximate a conic arc with a curve in two or three
  dimensional space. If two points are given, a straight line is
  produced, if three an approximation of a circular arc, and if four or
  five a conic arc.
  The output will be represented as a B-spline curve.
\end{minipg1} \\ \\
SYNOPSIS\\
        \>void s1611(\begin{minipg3}
                {\fov point}, {\fov numpt}, {\fov dim}, {\fov typept}, {\fov open}, {\fov order},
                {\fov startpar}, {\fov epsge}, {\fov endpar}, {\fov curve}, {\fov stat})
                \end{minipg3}\\[0.3ex]
                \>\>    double  \>      {\fov point}[\,];\\
                \>\>    int     \>      {\fov numpt};\\
                \>\>    int     \>      {\fov dim};\\
                \>\>    double  \>      {\fov typept}[\,];\\
                \>\>    int     \>      {\fov open};\\
                \>\>    int     \>      {\fov order};\\
                \>\>    double  \>      {\fov startpar};\\
                \>\>    double  \>      {\fov epsge};\\
                \>\>    double  \>      *{\fov endpar};\\
                \>\>    SISLCurve       \>      **{\fov curve};\\
                \>\>    int     \>      *{\fov stat};\\
\\
ARGUMENTS\\
        \>Input Arguments:\\
        \>\>    {\fov point}    \> - \>
                \begin{minipg2}
                  Array of length $dim\times numpt$ containing the
                  points/ derivatives to be interpolated.
                \end{minipg2}\\[0.3ex]
        \>\>    {\fov numpt}    \> - \>
                                No. of points/derivatives in the
                                point array.
                                \\
        \>\>    {\fov dim}      \> - \> \begin{minipg2}
                                The dimension of the space in which
                                the points lie.
                                \end{minipg2}\\
        \>\>    {\fov typept}   \> - \> \begin{minipg2}
                                Array (length numpt) containing type
                                indicator for points/derivatives/
                                second-derivatives:
                                \end{minipg2} \\[0.3ex]
                \>\>\>\>\>      1 : Ordinary point.\\
                \>\>\>\>\>      3 : Derivative to next point.\\
                \>\>\>\>\>      4 : Derivative to prior point.\\
        \>\>    {\fov open}     \> - \> Open or closed curve:\\
                \>\>\>\>\>      0 :     Closed curve, not implemented.\\
                \>\>\>\>\>      1 :     Open curve.\\
        \>\>    {\fov order}    \> - \> \begin{minipg2}
                                The order of the B-spline curve
                                to be produced.
                                \end{minipg2}\\
        \>\>    {\fov startpar}\> - \>  \begin{minipg2}
                                Parameter-value to be used at the
                                start of the curve.
                                \end{minipg2}\\
        \>\>    {\fov epsge}    \> - \> The geometry resolution.\\
\\
\newpagetabs
        \>Output Arguments:\\
        \>\>    {\fov endpar}   \> - \> \begin{minipg2}
                                Parameter-value used at the end
                                of the curve.
                                \end{minipg2}\\
        \>\>    {\fov curve}    \> - \> Pointer to the output B-spline curve.\\
        \>\>    {\fov stat}     \> - \> Status messages\\
                \>\>\>\>\>              $> 0$   : warning\\
                \>\>\>\>\>              $= 0$   : ok\\
                \>\>\>\>\>              $< 0$   : error\\
\\
NOTE\\
\>\begin{minipg6}
When four points/tangents are given as input, the xy term of the
implicit equation is set to zero. Thus the points might end on two
branches of a hyperbola and a straight line is produced. When
four or five points/tangents are given only three of these should
actually be points.
\end{minipg6}
\\ \\
EXAMPLE OF USE\\
                \>      \{ \\
                \>\>    double  \>      {\fov point}[30];\\
                \>\>    int     \>      {\fov numpt} = 10;\\
                \>\>    int     \>      {\fov dim} = 3;\\
                \>\>    double  \>      {\fov typept}[10];\\
                \>\>    int     \>      {\fov open};\\
                \>\>    int     \>      {\fov order};\\
                \>\>    double  \>      {\fov startpar};\\
                \>\>    double  \>      {\fov epsge};\\
                \>\>    double  \>      {\fov endpar};\\
                \>\>    SISLCurve       \>      *{\fov curve};\\
                \>\>    int     \>      {\fov stat};\\
                \>\>    \ldots \\
        \>\>s1611(\begin{minipg4}
                {\fov point}, {\fov numpt}, {\fov dim}, {\fov typept}, {\fov open}, {\fov order}, {\fov startpar}, {\fov epsge},\linebreak
                \&{\fov endpar}, \&{\fov curve}, \&{\fov stat});
                        \end{minipg4}\\
                \>\>    \ldots \\
                \>      \} \\
\end{tabbing}

\pgsbreak
\subsection{Compute a curve using the input points as
controlling \mbox{vertices}, automatic parameterization.}
\funclabel{s1630}
\begin{minipg1}
  To compute a curve using the input points as controlling
  vertices. The distances between the points are used as
  parametrization.
  The output will be represented as a B-spline curve.
\end{minipg1}\\ \\
SYNOPSIS\\
        \>void s1630(\begin{minipg3}
          {\fov epoint}, {\fov inbpnt}, {\fov astpar}, {\fov iopen}, {\fov idim}, {\fov ik}, {\fov rc}, {\fov jstat})
        \end{minipg3}\\[0.3ex]
        \>\> double \>  {\fov epoint}[\,];\\
        \>\> int    \>  {\fov inbpnt};\\
        \>\> double \>  {\fov astpar};\\
        \>\> int    \>  {\fov iopen};\\
        \>\> int    \>  {\fov idim};\\
        \>\> int    \>  {\fov ik};\\
        \>\> SISLCurve \> **{\fov rc};\\
        \>\> int    \>  *{\fov jstat};\\
\\
ARGUMENTS\\
        \>Input Arguments:\\
        \>\>    {\fov epoint} \> - \>
        \begin{minipg2}
          The array containing the points to be used as
          controlling vertices of the B-spline curve.
        \end{minipg2}\\[0.8ex]
        \>\>    {\fov inbpnt} \> - \> No. of points in epoint.\\
        \>\>    {\fov astpar} \> - \>
        \begin{minipg2}
          Parameter value to be used at the start of the curve.
        \end{minipg2}\\[0.8ex]
        \>\>    {\fov iopen} \> - \>
        \begin{minipg2}
          Open/closed/periodic condition.
        \end{minipg2}\\[0.8ex]
            \>\>\>\> $=-1$ \> : Closed and periodic.\\
            \>\>\>\> $=0$  \> : Closed.\\
            \>\>\>\> $=1$  \> : Open.\\
        \>\>    {\fov idim} \> - \> The dimension of the space.\\
        \>\>    {\fov ik}   \> - \> The order of the spline curve to be produced.\\
\\
        \>Output Arguments:\\
        \>\>    {\fov rc} \> - \> Pointer to the B-spline curve.\\
        \>\>    {\fov jstat}\> - \> Status message\\
                \>\>\>\>\> $< 0$ : Error.\\
                \>\>\>\>\> $= 0$ : Ok.\\
                \>\>\>\>\> $> 0$ : Warning.\\
\newpagetabs
EXAMPLE OF USE\\
        \>      \{ \\
        \>\>    double \> {\fov epoint}[30];\\
        \>\>    int    \> {\fov inbpnt} = 10;\\
        \>\>    double \> {\fov astpar} = 0.0;\\
        \>\>    int    \> {\fov iopen} = 1;\\
        \>\>    int    \> {\fov idim} = 3;\\
        \>\>    int    \> {\fov ik} = 4;\\
        \>\>    SISLCurve \> *{\fov rc} = NULL;\\
        \>\>    int    \> {\fov jstat};\\
        \>\>    \ldots \\
        \>\>s1630(\begin{minipg4}
          {\fov epoint}, {\fov inbpnt}, {\fov astpar}, {\fov iopen}, {\fov idim}, {\fov ik}, \&{\fov rc}, \&{\fov jstat});
        \end{minipg4}\\
        \>\>    \ldots \\
        \>      \}
\end{tabbing}

\pgsbreak
\subsection{Approximate the offset of a curve with a curve.}
\funclabel{s1360}
\begin{minipg1}
  To create a approximation of the offset to a curve within a
  tolerance.
  The output will be represented as a B-spline curve.\\
  With an offset of zero, this routine can be used to approximate any
  NURBS curve, within a tolerance, with a (non-rational) B-spline curve.
\end{minipg1} \\ \\
SYNOPSIS\\
        \>void s1360(\begin{minipg3}
        {\fov oldcurve}, {\fov offset}, {\fov epsge}, {\fov norm}, {\fov max}, {\fov dim}, {\fov newcurve}, {\fov stat})
                \end{minipg3}\\[0.3ex]
                \>\>    SISLCurve       \>      *{\fov oldcurve};\\
                \>\>    double  \>      {\fov offset};\\
                \>\>    double  \>      {\fov epsge};\\
                \>\>    double  \>      {\fov norm}[\,];\\
                \>\>    double  \>      {\fov max};\\
                \>\>    int     \>      {\fov dim};\\
                \>\>    SISLCurve       \>      **{\fov newcurve};\\
                \>\>    int     \>      *{\fov stat};\\
\\
ARGUMENTS\\
        \>Input Arguments:\\
        \>\>    {\fov oldcurve}\> - \> The input curve.\\
        \>\>    {\fov offset}   \> - \> \begin{minipg2}
                        The offset distance. If dim=2, a positive sign on
                        this value put the offset on the side of the positive
                        normal vector, and a negative sign puts the offset on
                        the negative normal vector. If dim=3, the offset direction is
                        determined by the cross product of the tangent
                        vector and the normal vector. The offset distance is
                        multiplied by this cross product.
                                \end{minipg2}\\[0.8ex]
        \>\>    {\fov epsge}    \> - \> \begin{minipg2}
                        Maximal deviation allowed between the true offset
                        curve and the approximated offset curve.
                                \end{minipg2}\\[0.3ex]
        \>\>    {\fov norm}     \> - \> Vector used in 3D calculations.\\
        \>\>    {\fov max}      \> - \> \begin{minipg2}
                        Maximal step length. It is neglected if
                        max$\leq$epsge. If max=0.0, then a maximal step
                        equal to the longest box side of the curve is used.
                                \end{minipg2}\\[0.8ex]
        \>\>    {\fov dim}      \> - \>The dimension of the space must be 2 or 3.\\
\\
NOTE\\
\>\begin{minipg6}
  If the vector norm and the curve tangent are parallel at some point,
  then the curve produced will not be an offset at this point, and it
  will probably move from one side of the input curve to the other side.
\end{minipg6}\\
\newpagetabs
        \>Output Arguments:\\
        \>\>    {\fov newcurve}\> - \> \begin{minipg2}
                        Pointer to the B-spline curve
                        approximating the offset curve.
                                       \end{minipg2}\\[0.8ex]
        \>\>    {\fov stat}     \> - \> Status messages.\\
                \>\>\>\>\>              $> 0$   : Warning.\\
                \>\>\>\>\>              $= 0$   : Ok.\\
                \>\>\>\>\>              $< 0$   : Error.\\
\\
EXAMPLE OF USE\\
                \>      \{ \\
                \>\>    SISLCurve       \>      *{\fov oldcurve};\\
                \>\>    double  \>      {\fov offset};\\
                \>\>    double  \>      {\fov epsge};\\
                \>\>    double  \>      {\fov norm}[3];\\
                \>\>    double  \>      {\fov max};\\
                \>\>    int     \>      {\fov dim} = 3;\\
                \>\>    SISLCurve       \>      *{\fov newcurve};\\
                \>\>    int     \>      {\fov stat};\\
                \>\>    \ldots \\
        \>\>s1360(\begin{minipg4}
        {\fov oldcurve}, {\fov offset}, {\fov epsge}, {\fov norm}, {\fov max}, {\fov dim}, \&{\fov newcurve}, \&{\fov stat});
                        \end{minipg4}\\
                \>\>    \ldots \\
                \>      \} \\
\end{tabbing}

\pgsbreak
\subsection{Approximate a curve with a sequence of straight lines.}
\funclabel{s1613}
\begin{minipg1}
  To calculate a set of points on a curve. The straight lines between the
  points will not deviate more than {\fov epsge} from the curve at any
  point.  The generated points will have the same spatial dimension as
  the input curve.
\end{minipg1} \\ \\
SYNOPSIS\\
        \>void s1613(\begin{minipg3}
        {\fov curve}, {\fov epsge}, {\fov points}, {\fov numpoints}, {\fov stat})
                \end{minipg3}\\[0.3ex]
                \>\>    SISLCurve       \>      *{\fov curve};\\
                \>\>    double  \>      {\fov epsge};\\
                \>\>    double  \>      **{\fov points};\\
                \>\>    int     \>      *{\fov numpoints};\\
                \>\>    int     \>      *{\fov stat};\\
\\
ARGUMENTS\\
        \>Input Arguments:\\
        \>\>    {\fov curve}    \> - \> \begin{minipg2}
                                The input curve.
                                \end{minipg2}\\
        \>\>    {\fov epsge}\> - \>     \begin{minipg2}
                                Geometry resolution, maximum distance allowed
                                between the curve and the straight lines that are to be
                                calculated.
                                \end{minipg2}\\[0.8ex]
        \>Output Arguments:\\
        \>\>    {\fov points}   \> - \> \begin{minipg2}
                                Calculated points,\\
                                (a vector of
                                $numpoints\times curve${\tt ->}$idim$ elements).
                                \end{minipg2}\\[0.3ex]
        \>\>    {\fov numpoints}\> - \>\begin{minipg2}
                                Number of calculated points.
                                \end{minipg2}\\
        \>\>    {\fov stat}     \> - \> Status messages\\
                \>\>\>\>\>              $> 0$   : warning\\
                \>\>\>\>\>              $= 0$   : ok\\
                \>\>\>\>\>              $< 0$   : error\\
EXAMPLE OF USE\\
                \>      \{ \\
                \>\>    SISLCurve       \>      *{\fov curve};\\
                \>\>    double  \>      {\fov epsge};\\
                \>\>    double  \>      *{\fov points};\\
                \>\>    int     \>      {\fov numpoints};\\
                \>\>    int     \>      {\fov stat};\\
                \>\>    \ldots \\
        \>\>s1613(\begin{minipg4}
                {\fov curve}, {\fov epsge}, \&{\fov points}, \&{\fov numpoints}, \&{\fov stat});
                        \end{minipg4}\\
                \>\>    \ldots \\
                \>      \}
\end{tabbing}

\pgsbreak
\section{Mirror a Curve}
\funclabel{s1600}
\begin{minipg1}
  To mirror a curve around a plane.
\end{minipg1} \\ \\
SYNOPSIS\\
        \>void s1600(\begin{minipg3}
                {\fov oldcurve}, {\fov point}, {\fov normal}, {\fov dim}, {\fov newcurve}, {\fov stat})
                \end{minipg3}\\[0.3ex]
                \>\>    SISLCurve       \>      *{\fov oldcurve};\\
                \>\>    double  \>      {\fov point}[\,];\\
                \>\>    double  \>      {\fov normal}[\,];\\
                \>\>    int     \>      {\fov dim};\\
                \>\>    SISLCurve       \>      **{\fov newcurve};\\
                \>\>    int     \>      *{\fov stat};\\
\\
ARGUMENTS\\
        \>Input Arguments:\\
        \>\>    {\fov oldcurve}\> - \>  Pointer to original curve.\\
        \>\>    {\fov point}    \> - \> A point in the plane.\\
        \>\>    {\fov normal}   \> - \> Normal vector to the plane.\\
        \>\>    {\fov dim}      \> - \> The dimension of the space.\\
\\
        \>Output Arguments:\\
        \>\>    {\fov newcurve}\> - \>  Pointer to the mirrored curve.\\
        \>\>    {\fov stat}     \> - \> Status messages\\
                \>\>\>\>\>              $> 0$   : warning\\
                \>\>\>\>\>              $= 0$   : ok\\
                \>\>\>\>\>              $< 0$   : error\\
\\
EXAMPLE OF USE\\
                \>      \{ \\
                \>\>    SISLCurve       \>      *{\fov oldcurve};\\
                \>\>    double  \>      {\fov point}[3];\\
                \>\>    double  \>      {\fov normal}[3];\\
                \>\>    int     \>      {\fov dim} = 3;\\
                \>\>    SISLCurve       \>      *{\fov newcurve};\\
                \>\>    int     \>      {\fov stat};\\
                \>\>    \ldots \\
        \>\>s1600(\begin{minipg4}
                {\fov oldcurve}, {\fov point}, {\fov normal}, {\fov dim}, \&{\fov newcurve}, \&{\fov stat});
                        \end{minipg4}\\
                \>\>    \ldots \\
                \>      \} \\
\end{tabbing}

\pgsbreak
\section{Conversion}
\subsection{Convert a curve of order up to four, to a sequence of cubic polynomials.}
\funclabel{s1389}
\begin{minipg1}
  Convert a curve of order up to 4 to a sequence of non-rational cubic
  segments with uniform parameterization.
\end{minipg1} \\
SYNOPSIS\\
        \>void s1389(\begin{minipg3}
                                {\fov curve}, {\fov cubic}, {\fov numcubic}, {\fov dim}, {\fov stat})
                \end{minipg3}\\[0.3ex]

                \>\>    SISLCurve       \>      *{\fov curve};\\
                \>\>    double  \>      **{\fov cubic};\\
                \>\>    int     \>      *{\fov numcubic};\\
                \>\>    int     \>      *{\fov dim};\\
                \>\>    int     \>      *{\fov stat};\\
ARGUMENTS\\
        \>Input Arguments:\\
        \>\>    {\fov curve}    \> - \> \begin{minipg2}
                                Pointer to the curve that is to be converted
                                \end{minipg2}\\[0.8ex]
        \>Output Arguments:\\
        \>\>    {\fov cubic}    \> - \> \begin{minipg2}
                                Array containing the sequence of cubic segments.
                                Each segment is represented by the start point,
                                followed by the start tangent, end point and end
                                tangent. Each segment needs 4*dim doubles for storage.
                                \end{minipg2}\\[0.3ex]
        \>\>    {\fov numcubic}\> - \>  \begin{minipg2}
                                Number of elements of length
                                (4*dim) in the array cubic
                                \end{minipg2}\\[0.8ex]
        \>\>    {\fov dim}      \> - \> \begin{minipg2}
                                The dimension of the geometric space.
                                \end{minipg2}\\
        \>\>    {\fov stat}     \> - \> Status messages\\
                \>\>\>\>\>      $> 0$   :\>\begin{minipg5}
                                                warning
                                        \end{minipg5}\\
                \>\>\>\>\>      $= 0$   :\> ok\\
                \>\>\>\>\>      $< 0$   :\> error\\
EXAMPLE OF USE\\
                \>      \{ \\
                \>\>    SISLCurve       \>      *{\fov curve};\\
                \>\>    double  \>      *{\fov cubic};\\
                \>\>    int     \>      {\fov numcubic};\\
                \>\>    int     \>      {\fov dim};\\
                \>\>    int     \>      {\fov stat};\\
                \>\>    \ldots \\
        \>\>s1389(\begin{minipg4}
                {\fov curve}, \&{\fov cubic}, \&{\fov numcubic}, \&{\fov dim},
                 \&{\fov stat});
                        \end{minipg4}\\
                \>\>    \ldots \\
                \>      \}
\end{tabbing}

\pgsbreak
\subsection{Convert a curve to a sequence of Bezier curves.}
\funclabel{s1730}
\begin{minipg1}
  To convert a curve to a sequence of Bezier curves. The Bezier
  curves are stored as one curve with all knots of multiplicity
  newcurve-$>$ik (order of the curve).
  If the input curve is rational, the generated Bezier curves will be
  rational too (i.e.\ there will be rational weights in the
  representation of the Bezier curves).
\end{minipg1} \\ \\
SYNOPSIS\\
        \>void s1730(\begin{minipg3}
        {\fov curve}, {\fov newcurve}, {\fov stat})
                \end{minipg3}\\[0.3ex]
                \>\>    SISLCurve       \>      *{\fov curve};\\
                \>\>    SISLCurve       \>      **{\fov newcurve};\\
                \>\>    int     \>      *{\fov stat};\\
\\
ARGUMENTS\\
        \>Input Arguments:\\
        \>\>    {\fov curve}    \> - \> The curve to convert.\\
\\
        \>Output Arguments:\\
        \>\>    {\fov newcurve}\> - \>\begin{minipg2}
                                The new curve
                                containing all
                                the Bezier curves.
                                \end{minipg2}\\[0.8ex]
        \>\>    {\fov stat}     \> - \> Status messages\\
                \>\>\>\>\>              $> 0$   : warning\\
                \>\>\>\>\>              $= 0$   : ok\\
                \>\>\>\>\>              $< 0$   : error\\
\\
EXAMPLE OF USE\\
                \>      \{ \\
                \>\>    SISLCurve       \>      *{\fov curve};\\
                \>\>    SISLCurve       \>      *{\fov newcurve};\\
                \>\>    int     \>      {\fov stat};\\
                \>\>    \ldots \\
        \>\>s1730(\begin{minipg4}
                {\fov curve}, \&{\fov newcurve}, \&{\fov stat});
                        \end{minipg4}\\
                \>\>    \ldots \\
                \>      \}
\end{tabbing}

\pgsbreak
\subsection{Pick out the next Bezier curve from a curve.}
\funclabel{s1732}
\begin{minipg1}
  To pick out the next Bezier curve from a curve. This function requires a
  curve represented as the curve that is output from s1730().
  If the input curve is rational, the generated Bezier curves will be
  rational too (i.e.\ there will be rational weights in the
  representation of the Bezier curves).
\end{minipg1} \\ \\
SYNOPSIS\\
        \>void s1732(\begin{minipg3}
        {\fov curve}, {\fov number}, {\fov startpar}, {\fov endpar}, {\fov coef}, {\fov stat})
                \end{minipg3}\\[0.3ex]
                \>\>    SISLCurve       \>      *{\fov curve};\\
                \>\>    int     \>      {\fov number};\\
                \>\>    double  \>      *{\fov startpar};\\
                \>\>    double  \>      *{\fov endpar};\\
                \>\>    double  \>      {\fov coef}[\,];\\
                \>\>    int     \>      *{\fov stat};\\
\\
ARGUMENTS\\
        \>Input Arguments:\\
        \>\>    {\fov curve}    \> - \> curve to pick from.\\
        \>\>    {\fov number}   \> - \>\begin{minipg2}
                                The number of the Bezier curve that is
                                to be picked, where $0\leq number<in/ik$
                                (i.e.\ the number of vertices in the
                                curve divided by the order of the curve).
                                \end{minipg2}\\[0.8ex]
\\
        \>Output Arguments:\\
        \>\>    {\fov startpar}\> - \>\begin{minipg2}
                                The start parameter value of the Bezier curve.
                                \end{minipg2}\\
        \>\>    {\fov endpar} \> - \>\begin{minipg2}
                                The end parameter value of the Bezier curve.
                                \end{minipg2}\\
        \>\>    {\fov coef}     \> - \>\begin{minipg2}
                                The vertices of the Bezier curve.
                                Space of size $(idim+1)\times ik$ (i.e.\
                                spatial dimension of curve $+1$ times the
                                order of the curve) must be allocated
                                outside the function.
                                \end{minipg2}\\[0.8ex]
        \>\>    {\fov stat}     \> - \> Status messages\\
                \>\>\>\>\>              $> 0$   : warning\\
                \>\>\>\>\>              $= 0$   : ok\\
                \>\>\>\>\>              $< 0$   : error\\
\newpagetabs
EXAMPLE OF USE\\
                \>      \{ \\
                \>\>    SISLCurve       \>      *{\fov curve};\\
                \>\>    int     \>      {\fov number};\\
                \>\>    double  \>      {\fov startpar};\\
                \>\>    double  \>      {\fov endpar};\\
                \>\>    double  \>      {\fov coef}[12];\\
                \>\>    int     \>      {\fov stat};\\
                \>\>    \ldots \\
        \>\>s1732(\begin{minipg4}
                {\fov curve}, {\fov number}, \&{\fov startpar}, \&{\fov endpar}, {\fov coef}, \&{\fov stat});
                        \end{minipg4}\\
                \>\>    \ldots \\
                \>      \}
\end{tabbing}

\pgsbreak
\subsection{Express a curve using a higher order basis.}
\funclabel{s1750}
\begin{minipg1}
  To describe a curve using a higher order basis.
\end{minipg1} \\ \\
SYNOPSIS\\
        \>void s1750(\begin{minipg3}
        {\fov curve}, {\fov order}, {\fov newcurve}, {\fov stat})
                \end{minipg3}\\[0.3ex]
                \>\>    SISLCurve       \>      *{\fov curve};\\
                \>\>    int     \>      {\fov order};\\
                \>\>    SISLCurve       \>      **{\fov newcurve};\\
                \>\>    int     \>      *{\fov stat};\\
\\
ARGUMENTS\\
        \>Input Arguments:\\
        \>\>    {\fov curve}    \> - \> The input curve.\\
        \>\>    {\fov order}    \> - \> \begin{minipg2}
                                Order of the new curve.
                                \end{minipg2}\\
\\
        \>Output Arguments:\\
        \>\>    {\fov newcurve}\> - \>\begin{minipg2}
                                The new curve of higher order.
                                \end{minipg2}\\
        \>\>    {\fov stat}     \> - \> Status messages\\
                \>\>\>\>\>              $> 0$   : warning\\
                \>\>\>\>\>              $= 0$   : ok\\
                \>\>\>\>\>              $< 0$   : error\\
\\
EXAMPLE OF USE\\
                \>      \{ \\
                \>\>    SISLCurve       \>      *{\fov curve};\\
                \>\>    double  \>      {\fov order};\\
                \>\>    SISLCurve       \>      *{\fov newcurve};\\
                \>\>    int     \>      stat;\\
                \>\>    \ldots \\
        \>\>s1750(\begin{minipg4}
                {\fov curve}, {\fov order}, \&{\fov newcurve}, \&{\fov stat});
                        \end{minipg4}\\
                \>\>    \ldots \\
                \>      \}
\end{tabbing}

\pgsbreak
\subsection{Express the ``i''-th derivative of an open curve as a curve.}
\funclabel{s1720}
\begin{minipg1}
To express the ``i''-th derivative of an open curve as a curve.
\end{minipg1} \\ \\
SYNOPSIS\\
        \>void s1720(\begin{minipg3}
        {\fov curve}, {\fov derive}, {\fov newcurve}, {\fov stat})
                \end{minipg3}\\[0.3ex]
                \>\>    SISLCurve       \>      *{\fov curve};\\
                \>\>    int     \>      {\fov derive};\\
                \>\>    SISLCurve       \>      **{\fov newcurve};\\
                \>\>    int     \>      *{\fov stat};\\
\\
ARGUMENTS\\
        \>Input Arguments:\\
        \>\>    {\fov curve}    \> - \> Curve to be differentiated.\\
        \>\>    {\fov derive}   \> - \> \begin{minipg2}
                                The order "i" of the derivative, where
                                $0 \leq derive $.
                                \end{minipg2}\\
\\
        \>Output Arguments:\\
        \>\>    {\fov newcurve}\> - \>\begin{minipg2}
                                The "i"-th derivative of a curve
                                represented as a curve.
                                \end{minipg2}\\[0.8ex]
        \>\>    {\fov stat}     \> - \> Status messages\\
                \>\>\>\>\>              $> 0$   : warning\\
                \>\>\>\>\>              $= 0$   : ok\\
                \>\>\>\>\>              $< 0$   : error\\
\\
EXAMPLE OF USE\\
                \>      \{ \\
                \>\>    SISLCurve       \>      *{\fov curve};\\
                \>\>    int     \>      {\fov derive};\\
                \>\>    SISLCurve       \>      *{\fov newcurve};\\
                \>\>    int     \>      {\fov stat};\\
                \>\>    \ldots \\
        \>\>s1720(\begin{minipg4}
                {\fov curve}, {\fov derive}, \&{\fov newcurve}, \&{\fov stat});
                        \end{minipg4}\\
                \>\>    \ldots \\
                \>      \}
\end{tabbing}

\pgsbreak
\subsection{Express a 2D or 3D ellipse as a curve.}
\funclabel{s1522}
\begin{minipg1}
  Convert a 2D or 3D analytical ellipse to a curve.
  The curve will be geometrically exact.
\end{minipg1} \\ \\
SYNOPSIS\\
        \> void s1522(\begin{minipg3}
            {\fov normal},  {\fov centre},  {\fov ellipaxis},  {\fov ratio},  {\fov dim},  {\fov ellipse},  {\fov jstat})
                \end{minipg3}\\[0.3ex]
                \>\>    double \> {\fov normal}[\,];\\
                \>\>    double \> {\fov centre}[\,];\\
                \>\>    double \> {\fov ellipaxis}[\,];\\
                \>\>    double \> {\fov ratio};\\
                \>\>    int    \> {\fov dim};\\
                \>\>    SISLCurve \> **{\fov ellipse};\\
                \>\>    int    \> *{\fov jstat};\\
\\
ARGUMENTS\\
        \>Input Arguments:\\
        \>\>    {\fov normal} \> - \>
        \begin{minipg2}
          3D normal to ellipse plane (not necessarily normalized).  Used
          if $dim=3$.
        \end{minipg2}\\[0.8ex]
        \>\>    {\fov centre} \> - \>
        \begin{minipg2}
          Centre of ellipse (2D if $dim=2$ and 3D if $dim=3$).
        \end{minipg2}\\[0.8ex]
        \>\>    {\fov ellipaxis} \> - \>
        \begin{minipg2}
          This will be used as starting point
          for the ellipse curve (2D if $dim=2$ and 3D if $dim=3$).
        \end{minipg2}\\[0.8ex]
        \>\>    {\fov ratio} \> - \>
        \begin{minipg2}
          The ratio between the length of the given ellipaxis and the
          length of the other axis, i.e. $|ellipaxis| / |other axis|$
          (a compact representation format).
        \end{minipg2}\\[0.8ex]
        \>\>    {\fov dim}\> - \>
        \begin{minipg2}
          Dimension of the space in which the elliptic nurbs curve lies (2 or 3).
        \end{minipg2}\\[0.8ex]
\\
        \>Output Arguments:\\
        \>\>    {\fov ellipse} \> - \>
        \begin{minipg2}
          Ellipse curve (2D if $dim=2$ and 3D if $dim=3$).
        \end{minipg2}\\[0.8ex]
        \>\>    {\fov stat}     \> - \> Status messages\\
                \>\>\>\>\>              $> 0$   : warning\\
                \>\>\>\>\>              $= 0$   : ok\\
                \>\>\>\>\>              $< 0$   : error\\
\newpagetabs
EXAMPLE OF USE\\
        \>      \{ \\
        \>\>    double \> {\fov normal}[3];\\
        \>\>    double \> {\fov centre}[3];\\
        \>\>    double \> {\fov ellipaxis}[3];\\
        \>\>    double \> {\fov ratio};\\
        \>\>    int    \> {\fov dim} = 3;\\
        \>\>    SISLCurve \> *{\fov ellipse};\\
        \>\>    int    \> {\fov jstat};\\
        \>\>    \ldots \\
        \>\>s1522(\begin{minipg4}
          {\fov normal},  {\fov centre},  {\fov ellipaxis},  {\fov ratio},  {\fov dim},  \&{\fov ellipse},  \&{\fov jstat});
        \end{minipg4}\\
        \>\>    \ldots \\
        \>      \}
\end{tabbing}

\pgsbreak
\subsection{Express a conic arc as a curve.}
\funclabel{s1011}
\begin{minipg1}
  Convert an analytic conic arc to a curve.
  The curve will be geometrically exact.
  The arc is given by position at start, shoulder point and end, and a
  shape factor.
\end{minipg1} \\ \\
SYNOPSIS\\
        \>void s1011(\begin{minipg3}
          {\fov start\_pos},  {\fov top\_pos},  {\fov end\_pos},  {\fov shape},  {\fov dim},  {\fov arc\_seg},  {\fov stat})
        \end{minipg3}\\[0.3ex]
        \>\>    double \> {\fov start\_pos}[\,];\\
        \>\>    double \> {\fov top\_pos}[\,];\\
        \>\>    double \> {\fov end\_pos}[\,];\\
        \>\>    double \> {\fov shape};\\
        \>\>    int    \> {\fov dim};\\
        \>\>    SISLCurve \> **{\fov arc\_seg};\\
        \>\>    int    \> *{\fov stat};\\
\\
ARGUMENTS\\
        \>Input Arguments:\\
        \>\>    {\fov start\_pos} \> - \> Start point of segment.\\
        \>\>    {\fov top\_pos}   \> - \> \begin{minipg2}
                                            Shoulder point of
                                            segment. This is the
                                            intersection point of the
                                            tangents in {\fov start\_pos} and
                                            {\fov end\_pos}.
                                          \end{minipg2}\\[0.8ex]
        \>\>    {\fov end\_pos}   \> - \> End point of segment.\\
        \>\>    {\fov shape}      \> - \> Shape factor, must be $\geq 0$.\\
                \>\>\>\>\> $< 0.5$, an ellipse,\\
                \>\>\>\>\> $= 0.5$, a parabola,\\
                \>\>\>\>\> $> 0.5$, a hyperbola,\\
                \>\>\>\>\> $\geq 1$,
                           \begin{minipg5}
                             the start and end points lies on different branches of
                             the hyperbola.  We want a single arc
                             segment, therefore if $shape\geq 1$, shape
                             is set to $0.999999$.
                           \end{minipg5}\\[0.8ex]
        \>\>    {\fov dim}        \> - \> The spatial dimension of the curve to
                                          be produced.\\
\\
        \>Output Arguments:\\
        \>\>    {\fov jstat} \> - \> Status message\\
                \>\>\>\>\> $< 0$ : Error.\\
                \>\>\>\>\> $= 0$ : Ok.\\
                \>\>\>\>\> $> 0$ : Warning.\\
        \>\>    {\fov arc\_seg} \> - \> Pointer to the curve produced.\\
\newpagetabs
EXAMPLE OF USE\\
        \>      \{ \\
        \>\>    double \> {\fov start\_pos}[3];\\
        \>\>    double \> {\fov top\_pos}[3];\\
        \>\>    double \> {\fov end\_pos}[3];\\
        \>\>    double \> {\fov shape};\\
        \>\>    int    \> {\fov dim} = 3;\\
        \>\>    SISLCurve \> *{\fov arc\_seg};\\
        \>\>    int    \> {\fov stat};\\
        \>\>    \ldots \\
        \>\>s1011(\begin{minipg4}
        {\fov start\_pos},  {\fov top\_pos},  {\fov end\_pos},  {\fov shape},  {\fov dim},  \&{\fov arc\_seg},  \&{\fov stat});
      \end{minipg4}\\
      \>\>    \ldots \\
      \>      \}
\end{tabbing}

\pgsbreak
\subsection{Express a truncated helix as a curve.}
\funclabel{s1012}
\begin{minipg1}
  Convert an analytical truncated helix to a curve.
  The curve will be geometrically exact.
\end{minipg1}\\ \\
SYNOPSIS\\
        \>void s1012(\begin{minipg3}
          {\fov start\_pos}, {\fov axis\_pos}, {\fov axis\_dir}, {\fov frequency}, {\fov numb\_quad},
          {\fov counter\_clock}, {\fov helix}, {\fov stat})
        \end{minipg3}\\[0.3ex]
        \>\>    double \> {\fov start\_pos}[\,];\\
        \>\>    double \> {\fov axis\_pos}[\,];\\
        \>\>    double \> {\fov axis\_dir}[\,];\\
        \>\>    double \> {\fov frequency};\\
        \>\>    int    \> {\fov numb\_quad};\\
        \>\>    int    \> {\fov counter\_clock};\\
        \>\>    SISLCurve \> **{\fov helix};\\
        \>\>    int    \> *{\fov stat};\\
\\
ARGUMENTS\\
        \>Input Arguments:\\
        \>\>    {\fov start\_pos} \> - \> Start position on the helix.\\
        \>\>    {\fov axis\_pos}  \> - \> Point on the helix axis.\\
        \>\>    {\fov axis\_dir}  \> - \> Direction of the helix axis.\\
        \>\>    {\fov frequency}  \> - \> \begin{minipg2}
                                            The length along the helix
                                            axis for one period of revolution.
                                          \end{minipg2}\\[0.8ex]
        \>\>    {\fov numb\_quad} \> - \> Number of quadrants in the helix.\\
        \>\>    {\fov counter\_clock} \> - \> Flag for direction of revolution:\\
                \>\>\>\>\> $= 0$ : clockwise,\\
                \>\>\>\>\> $= 1$ : counter\_clockwise.\\
\\
        \>Output Arguments:\\
        \>\>    {\fov jstat} \> - \> Status message\\
                \>\>\>\>\> $< 0$ : Error.\\
                \>\>\>\>\> $= 0$ : Ok.\\
                \>\>\>\>\> $> 0$ : Warning.\\
        \>\>    {\fov helix} \> - \> Pointer to the helix curve produced.\\
\newpagetabs
EXAMPLE OF USE\\
        \>      \{ \\
        \>\>    double \> {\fov start\_pos}[3];\\
        \>\>    double \> {\fov axis\_pos}[3];\\
        \>\>    double \> {\fov axis\_dir}[3];\\
        \>\>    double \> {\fov frequency};\\
        \>\>    int    \> {\fov numb\_quad};\\
        \>\>    int    \> {\fov counter\_clock};\\
        \>\>    SISLCurve \> *{\fov helix};\\
        \>\>    int    \> {\fov stat};\\
        \>\>    \ldots \\
        \>\>s1012(\begin{minipg4}
          {\fov start\_pos}, {\fov axis\_pos}, {\fov axis\_dir}, {\fov frequency}, {\fov numb\_quad},
          {\fov counter\_clock}, \&{\fov helix}, \&{\fov stat})
        \end{minipg4}\\
        \>\>    \ldots \\
        \>      \}
\end{tabbing}

