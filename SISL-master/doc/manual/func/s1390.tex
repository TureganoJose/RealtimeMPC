\subsection{Compute a rectangular blending surface from a set of
\mbox{B-spline} input curves.}
\funclabel{s1390}
\begin{minipg1}
  Make a 4-edged blending surface between 4 B-spline (i.e.\ NOT
  rational) curves where each curve is associated with a number of
  cross-derivative B-spline (i.e.\ NOT rational) curves.
  The output is represented as a B-spline surface.
  The input curves are numbered successively around the blending
  parameter, and the directions of the curves are expected to be as
  follows when this routine is entered:
\begin{center}
\begin{picture}(180,110)(0,0)
        \put(50,15){\framebox(80,80)}
        \put(40,55){\vector(0,1){20}}
        \put(140,55){\vector(0,1){20}}
        \put(90,5){\vector(1,0){20}}
        \put(90,105){\vector(1,0){20}}
        \put(40,45){\makebox(0,0){4}}
        \put(140,45){\makebox(0,0){2}}
        \put(80,5){\makebox(0,0){1}}
        \put(80,105){\makebox(0,0){3}}

        \put(60,20){\vector(1,0){40}}
        \put(85,28){\makebox(0,0){$(i)$}}
        \put(55,25){\vector(0,1){40}}
        \put(65,50){\makebox(0,0){$(ii)$}}
\end{picture}\\
$(i) \; \; \;$ first parameter direction of the surface.\\
$(ii)$   second parameter direction of the surface.\\
\end{center}
        NB!     The cross-derivatives are always pointing into the patch, and note the
                directions in the above diagram.
\end{minipg1}\\ \\
SYNOPSIS\\
        \>void s1390(\begin{minipg3}
                {\fov curves}, {\fov surf}, {\fov numder}, {\fov stat})
                \end{minipg3}\\[0.3ex]
                \>\>    SISLCurve       \>      *{\fov curves}[\,];\\
                \>\>    SISLSurf        \>      **{\fov surf};\\
                \>\>    int     \>      {\fov numder}[\,];\\
                \>\>    int     \>      *{\fov stat};\\
\\
ARGUMENTS\\
        \>Input Arguments:\\
        \>\>    {\fov curves}   \> - \>
        \begin{minipg2}
          Pointers to the boundary B-spline curves:\\
          $curves[i], i=0,\ldots,numder[0]-1,$
          are pointers to position and cross-derivatives along the first edge.
        \end{minipg2}\\[0.3ex]
        \>\>\>\>
        \begin{minipg2}
          $curves[i],$\\
          $i=numder[0],\ldots,numder[0]+numder[1]-1,$
          are pointers to position  and cross-derivatives
          along the second edge.
        \end{minipg2}\\[0.3ex]
        \>\>\>\>
        \begin{minipg2}
          $curves[i], i=numder[0]+numder[1],\ldots,$\\
          $numder[0]+numder[1]+numder[2]-1,$
          are pointers to position and cross-derivatives
          along the third edge.
        \end{minipg2}\\[0.3ex]
\newpagetabs
        \>\>\>\>
        \begin{minipg2}
          $curves[i],$\\
          $i=numder[0]+numder[1]+numder[2],\ldots,$\\
          $numder[0]+numder[1]+numder[2]+numder[3]-1,$
          are  pointers to position
          and cross-derivatives along the fourth edge.
        \end{minipg2}\\[0.3ex]
        \>\>    {\fov numder}   \> - \> \begin{minipg2}
                                Array of length 4, numder[i] gives the
                                 number of curves on edge number $i+1$.
                                \end{minipg2}\\[0.3ex]
\\
        \>Output Arguments:\\
        \>\>    {\fov surf}\> - \>      \begin{minipg2}
                                Pointer to the blending B-spline surface.
                                \end{minipg2}\\
        \>\>    {\fov stat}     \> - \> Status messages\\
                \>\>\>\>\>              $> 0$   : warning\\
                \>\>\>\>\>              $= 0$   : ok\\
                \>\>\>\>\>              $< 0$   : error\\
\\
EXAMPLE OF USE\\
                \>      \{ \\
                \>\>    SISLCurve       \>      *{\fov curves}[8];\\
                \>\>    SISLSurf        \>      *{\fov surf};\\
                \>\>    int     \>      {\fov numder}[4];\\
                \>\>    int     \>      {\fov stat};\\
                \>\>    \ldots \\
        \>\>s1390(\begin{minipg4}
                {\fov curves}, \&{\fov surf}, {\fov numder}, \&{\fov stat})
                        \end{minipg4}\\
                \>\>    \ldots \\
                \>      \}
\end{tabbing}
