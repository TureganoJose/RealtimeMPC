\subsection{Data reduction: B-spline curve as input.}
\funclabel{s1940}
\begin{minipg1}
To remove as many knots as possible from a spline curve
 	   without perturbing the curve more than a given
 	   tolerance.
\end{minipg1} \\ \\
SYNOPSIS\\
        \> void s1940(\begin{minipg3}
            {\fov oldcurve},  {\fov eps},  {\fov startfix},  {\fov endfix},  {\fov iopen},  {\fov itmax},  {\fov newcurve},
       {\fov maxerr},  {\fov stat})
                \end{minipg3}\\
                \>\>    SISLCurve \> *{\fov oldcurve};\\
                \>\>    double \> {\fov eps}[\,];\\
                \>\>    int \> {\fov startfix};\\
                \>\>    int \> {\fov endfix};\\
                \>\>    int \> {\fov iopen};\\
                \>\>    int \> {\fov itmax};\\
                \>\>    SISLCurve \> **{\fov newcurve};\\
                \>\>    double \> {\fov maxerr}[\,];\\
                \>\>    int \> *{\fov stat};\\
\\
ARGUMENTS\\
	\>Input Arguments:\\
        \>\>    {\fov oldcurve}\> - \>  \begin{minipg2}
                     pointer to the original spline curve. 
                               \end{minipg2}\\
        \>\>    {\fov eps}\> - \>  \begin{minipg2}
                     double array giving the desired absolute accuracy
                        of the final approximation as compared to oldcurve.
                        If oldcurve is a spline curve in a space of
			 dimension dim, then eps must have length dim.
                        Note that it is not relative, but absolute accuracy
                        that is being used. This means that the difference
                        in component i at any parameter value, between
                        the given curve and the approximation, is to be
                        less than eps[i]. Note that in such comparisons
                        the same parametrization is used for both curves.
                               \end{minipg2}\\[0.8ex]
        \>\>    {\fov startfix}\> - \>  \begin{minipg2}
                     the number of derivatives to be kept fixed at
			 the beginning of the knot interval.
                     The $0,\ldots,(startfix-1)$ derivatives will be kept fixed.
                        If startfix $<0$, this routine will set it to 0.
                        If startfix $<$ the order of the curve, this routine
                        will set it to the order.
                               \end{minipg2}\\[0.8ex]
        \>\>    {\fov endfix}\> - \>  \begin{minipg2}
                     the number of derivatives to be kept fixed at
		         the end of the knot interval.
                     The $0,\ldots,(endfix-1)$ derivatives will be kept fixed.
                        If endfix $<0$, this routine will set it to 0.
                        If endfix $<$ the order of the curve, this routine
                        will set it to the order.
                               \end{minipg2}\\[0.8ex]
        \>\>    {\fov iopen}\> - \>  Open/closed parameter\\
            \>\>\>\>  $= 1$  : Produce open curve.\\
            \>\>\>\>  $= 0$ : Produce closed, non-periodic curve if possible.\\
            \>\>\>\>  $= -1$ : Produce closed, periodic curve if possible.\\
        \>\>    {\fov itmax}\> - \>  \begin{minipg2}
                     maximum number of iterations. The routine will
                        follow an iterative procedure trying to remove
                        more and more knots. The process will almost always
                        stop after less than 10 iterations and it will often
                        stop after less than 5 iterations. A suitable
                        value for itmax is therefore usually in the region
                        3-10.
                               \end{minipg2}\\
\\
	\>Output Arguments:\\
        \>\>    {\fov }\> - \>  \begin{minipg2}
        
                               \end{minipg2}\\
        \>\>    {\fov newcurve}\> - \>  \begin{minipg2}
                     the spline approximation on the reduced
                        knot vector.
                               \end{minipg2}\\
        \>\>    {\fov maxerr}\> - \>  \begin{minipg2}
                     double array containing an upper bound for the
			 pointwise error in each of the components of the
		         spline approximation. The two curves oldcurve and
			 newcurve are compared at the same parameter value,
                        i.e., if oldcurve is f and newcurve is g, then
			               $|f(t)-g(t)| <= eps$
			 in each of the components.
                               \end{minipg2}\\
        \>\>    {\fov stat}     \> - \> Status messages\\
                \>\>\>\>\>              $> 0$   : Warning.\\
                \>\>\>\>\>              $= 0$   : Ok.\\
                \>\>\>\>\>              $< 0$   : Error.\\
\\
EXAMPLE OF USE\\
		\>      \{ \\

                \>\>    SISLCurve \> *{\fov oldcurve};\\
                \>\>    double \> {\fov eps}[\,];\\
                \>\>    int \> {\fov startfix};\\
                \>\>    int \> {\fov endfix};\\
                \>\>    int \> {\fov iopen};\\
                \>\>    int \> {\fov itmax};\\
                \>\>    SISLCurve \> **{\fov newcurve};\\
                \>\>    double \> {\fov maxerr}[\,];\\
                \>\>    int \> *{\fov stat};\\                \>\>    \ldots \\
        \>\>s1940(\begin{minipg4}
            {\fov oldcurve},  {\fov eps},  {\fov startfix},  {\fov endfix},  {\fov iopen},  {\fov itmax},  {\fov newcurve},
       {\fov maxerr},  {\fov stat});
                \end{minipg4}\\
                \>\>    \ldots \\
		\>      \}
\end{tabbing}
