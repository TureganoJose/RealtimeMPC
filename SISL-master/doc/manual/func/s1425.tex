\subsection{Compute the position and the derivatives of a surface at a
given parameter value pair.}
\funclabel{s1425}
\begin{minipg1}
  To compute the value and $ider1\times ider2$ first derivatives
  of a tensor product surface at the point with parameter value
  ({\fov epar}[0], {\fov epar}[1]).
  The derivatives that will be computed are $D(i,j)$,
  $i=0,1,\dots,ider1$, $j=0,1,\dots,ider2$. The calculations are
  from the right hand or left hand side.
\end{minipg1} \\ \\
SYNOPSIS\\
        \>void s1425(\begin{minipg3}
          {\fov ps1}, {\fov ider1}, {\fov ider2}, {\fov iside1}, {\fov iside2}, {\fov epar}, {\fov ileft1}, {\fov ileft2}, {\fov eder}, {\fov jstat})
        \end{minipg3}\\
        \>\>    SISLSurf    \>  *{\fov ps1};\\
        \>\>    int    \>  {\fov ider1};\\
        \>\>    int    \>  {\fov ider2};\\
        \>\>    int    \>  {\fov iside1};\\
        \>\>    int    \>  {\fov iside2};\\
        \>\>    double \>  {\ epar}[\,];\\
        \>\>    int    \>  *{\fov ileft1};\\
        \>\>    int    \>  *{\fov ileft2};\\
        \>\>    double \>  {\fov eder}[\,];\\
        \>\>    int    \>  *{\fov jstat};\\
\\
ARGUMENTS\\
        \>Input Arguments:\\
        \>\>    {\fov ps1}\> - \>  \begin{minipg2}
                     Pointer to the surface for which position
                       and derivatives are to be computed.
                               \end{minipg2}\\[0.8ex]
        \>\>    {\fov ider1}\> - \>
        \begin{minipg2}
          The number of derivatives to be computed with respect
          to the first parameter direction.
        \end{minipg2}\\[0.8ex]
                     \>\>\>\> $<0$ \> \> : Error, no derivative calculated.\\
                     \>\>\>\> $=0$ \> \> :
                     \begin{minipg5}
                       No derivatives with respect to the first
                       parameter direction will be computed.
                       (Only derivatives of the type
                       $D(0,0),D(0,1),\dots,D(0,ider2)$).
                     \end{minipg5}\\[0.8ex]
                     \>\>\>\> $=1$ \> :
                     \begin{minipg5}
                       Derivatives up to first order with respect to
                       the first parameter direction will be
                       computed.
                     \end{minipg5}\\[0.8ex]
                     \>\>\>\>      \>   etc.\\
\newpagetabs
        \>\>    {\fov ider2}\> - \>  \begin{minipg2}
                     The number of derivatives to be computed with respect
                       to the second parameter direction.
        \end{minipg2}\\[0.8ex]
                     \>\>\>\> $<0$ \> \> : Error, no derivative calculated.\\
                     \>\>\>\> $=0$ \> \> :
                     \begin{minipg5}
                       No derivatives with respect to the second
                       parameter direction will be computed.
                       (Only derivatives of the type
                       $D(0,0),D(1,0),\dots,D(ider1,0)$).
                     \end{minipg5}\\[0.8ex]
                     \>\>\>\> $=1$ \> :
                     \begin{minipg5}
                       Derivatives up to first order with respect to
                       the second parameter direction will be computed.
                     \end{minipg5}\\[0.8ex]
                     \>\>\>\>      \>   etc.\\
        \>\>    {\fov iside1}\> - \>
        \begin{minipg2}
          Indicator telling if the derivatives in the first
          parameter direction is to be calculated from the
          left or from the right:
        \end{minipg2}\\[0.3ex]
                      \>\>\>\> $<0$ \> :
                      \begin{minipg5}
                        Calculate derivative from the left hand side.
                      \end{minipg5}\\[0.8ex]
                      \>\>\>\> $\geq 0$ \> :
                      \begin{minipg5}
                        Calculate derivative from the right hand side.
                      \end{minipg5}\\[0.8ex]
        \>\> {\fov iside2}\> - \>
        \begin{minipg2}
          Indicator telling if the derivatives in the second
          parameter direction is to be calculated from the
          left or from the right:
        \end{minipg2}\\[0.3ex]
                      \>\>\>\> $<0$ \> :
                      \begin{minipg5}
                        Calculate derivative from the left hand side.
                      \end{minipg5}\\[0.8ex]
                      \>\>\>\> $\geq 0$ \> :
                      \begin{minipg5}
                        Calculate derivative from the right hand side.
                      \end{minipg5}\\[0.8ex]
        \>\>    {\fov epar}\> - \>
        \begin{minipg2}
          Array of dimension 2 containing the
          parameter values of the point at which the position
          and derivatives are to be computed.
        \end{minipg2}\\[0.8ex]
\\
        \>Input/Output Arguments:\\
        \>\>    {\fov ileft1}\> - \>
        \begin{minipg2}
          Pointer to the interval in the knot vector
          in the first parameter direction where {\fov epar}[0]
          is located. If {\fov et1} is the knot vector in the first
          parameter direction, the relation
          \[
          et1[ileft]\leq epar[0] < et1[ileft+1],
          \]
          should hold. (If $epar[0]=et1[in1]$ then {\fov ileft} should
          be $in1-1$. Here {\fov in1} is the number of B-spline
          coefficients associated with {\fov et1}.)
          If {\fov ileft1} does not have the right value upon
          entry to the routine, its value will be changed
          to the value satisfying the above condition.
        \end{minipg2}\\[0.8ex]
        \>\>    {\fov ileft2}\> - \>
        \begin{minipg2}
          Pointer to the interval in the knot vector
          in the second parameter direction where {\fov epar}[1]
          is located. If {\fov et2} is the knot vector in the second
          parameter direction, the relation
          \[
          et2[ileft]\leq epar[1] < et2[ileft+1],
          \]
          should hold. (If $epar[1]=et2[in2]$ then {\fov ileft} should
          be $in2-1$. Here {\fov in2} is the number of B-spline
          coefficients associated with {\fov et2}.)
          If {\fov ileft2} does not have the right value upon
          entry to the routine, its value will be changed
          to the value satisfying the above condition.
        \end{minipg2}\\[0.8ex]
\\
        \>Output Arguments:\\
        \>\>    {\fov eder}\> - \>
        \begin{minipg2}
          Array of dimension $(ider2+1)*(ider1+1)*idim$
          containing the position and the derivative vectors
          of the surface at the point with parameter value
          ({\fov epar}[0], {\fov epar}[1]).
          ({\fov idim} is the number of components of each B-spline
          coefficient, i.e. the dimension of the Euclidean
          space in which the surface lies.)
          These vectors are stored in the following order:
          First the {\fov idim} components of the position vector,
          then the {\fov idim} components of the $D(1,0)$ vector,
          and so on up to the {\fov idim} components of the $D(ider1,0)$
          vector,
          then the {\fov idim} components of the $D(1,1)$ vector etc.
          Equivalently, if {\fov eder} is considered to be a
          three dimensional array, then its declaration in C
          would be $eder[ider2+1,ider1+1,idim]$.
        \end{minipg2}\\[0.8ex]
        \>\>    {\fov jstat}     \> - \> Status messages\\
                \>\>\>\>\>      $> 0$ : \>      Warning.\\
                \>\>\>\>\>      $= 0$ : \>      Ok.\\
                \>\>\>\>\>      $< 0$ : \>      Error.\\
\newpagetabs
EXAMPLE OF USE\\
        \>      \{ \\
        \>\>    SISLSurf    \>  *{\fov ps1};\\
        \>\>    int    \>  {\fov ider1} = 1;\\
        \>\>    int    \>  {\fov ider2} = 1;\\
        \>\>    int    \>  {\fov iside1};\\
        \>\>    int    \>  {\fov iside2};\\
        \>\>    double \>  {\fov epar}[2];\\
        \>\>    int    \>  {\fov ileft1} = 0;\\
        \>\>    int    \>  {\fov ileft2} = 0;\\
        \>\>    double \>  {\fov eder}[12];\\
        \>\>    int    \>  {\fov jstat} = 0;\\
        \>\>    \ldots \\
        \>\>s1425(\begin{minipg4}
          {\fov ps1}, {\fov ider1}, {\fov ider2}, {\fov iside1}, {\fov iside2}, {\fov epar}, \&{\fov ileft1}, \&{\fov ileft2}, {\fov eder}, \&{\fov jstat});
        \end{minipg4}\\
        \>\>    \ldots \\
        \>      \}
\end{tabbing}
